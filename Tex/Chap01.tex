\chapter{绪论}
\label{chap:introduction}

\section{研究背景及现实意义}
\subsection{医学影像分析的研究背景}

临床医学历经几百年的发展,传统的“视、触、叩、听”已经不能满足现代化医疗的诊断需求,医学影像极大地变革了传统的诊疗体系,成熟的成像模式不断完善,新技术不断涌现,在疾病筛查、早期诊断、治疗方案选择和预后评估等方面发挥着举足轻重的作用。用于医疗诊断的影像使医生能够更早地发现疾病并改善患者预后,介入或术中成像有助于消除和治愈许多检测到的疾病,能更早更有效地诊断身心健康状况,为临床诊疗提供了全面的视角和丰富的信息,迅速地被广泛应用于临床领域。目前临床医学已经无法离开医学影像,并且随着医学影像的发展,临床诊疗将越来越依赖于影像。

医学影像包含有两个主要组成部分:图像形成或重建:从数据到图像,以及图像处理或分析:从图像到图像(去噪等)以及从图像到特征(识别等)。医学影像自1895年伦琴发现X射线以来,综合利用物理中的各种物质波、光电子技术以及计算机技术,从宏观到微观,由静态到动态,由单模到多模,由2D到3D,形成了各种的成像模式,包括X射线(X-ray),超声(Ultrasound,US),计算机断层扫描(Computed Tomography,CT),磁共振成像(Magnetic Resonance Imaging,MRI),正电子断层扫描(Positron Emission Computed Tomography , PET ),以及内窥镜,病理切片等。

自从将医学图像扫描加载到计算机中,研究人员就致力于构建自动化分析系统。 最初从20世纪70年代到90年代,医学影像分析是通过处理低阶像素(边缘、线、滤波器和区域生长算法)和数学建模(拟合线,圆和椭圆)并应用复合规则系统解决特定任务。
在20世纪90年代末,随着大数据和计算机通讯技术的发展,影像归档和通信系统(Picture Archiving and Communication Systems,PACS)和医学数字成像和通信标准(Digital Imaging and Communications in Medicine,DICOM)等技术的成熟既解决了影像的采集问题,又解决了数据的传输和存储问题,而基于训练数据的监督机器学习在医学影像分析中越来越受欢迎,例如活动形状模型(用于分割),图谱(Atlas)方法(适合形成新的训练数据)以及特征提取和统计分类器的使用(用于计算机辅助检测和诊断),这种模式识别或机器学习方法现在仍然非常流行,并构成了许多成功的商用医学影像分析系统的基础。

近来,医学影像分析也从传统的计算机辅助检测发展到火热的影像组学(Radiomics)\citep{Lambin2015},它将影像内包含的所有信息提取出来然后进行综合系统化分析。更确切的说,影像组学是采用自动化算法从影像的感兴趣区内提取出大量的特征信息作为研究对象,并进一步采用多样化的统计分析和数据挖掘方法从高通量量信息中提取和剥离出真正起作用的关键信息,最终用于疾病的辅助诊断、分类或分级。

但当今世界医疗卫生系统每天都会浪费大量的资源和时间,对医学影像内容的错误理解会造成错误诊断,导致很多不必要的额外检查,导致治疗计划的延迟,大大减少了如果早期正确发现的生存率或缓解率。同时不同的影像质量和不同的工作流程会导致临床上医生对影像内容的理解具有很大的主观性,对于通常在临床筛查和风险评估时所获得的医学图像,这些结构可能具有相当模糊的边界和低对比度,即使对于有经验的临床医生来说,图像级解释也是具有挑战性和耗时的任务。结合机器学习和计算机视觉的人工智能(Artificial Intelligence,AI)算法被用来帮助临床医生,提高医师对患者影像的理解,从而改善诊断,治疗以及由此产生的预后效果,将作为一个固化已有经验的临床助理,给临床医生的工作方式带来转变,显著提高工作流程效率,而不增加临床医生的负担。
\subsection{课题研究意义}

人工智能进入医学影像领域,主要为解决当前医学影像分析面临误诊率高、医师缺口大的问题,其一,如何快速自动准确的分析日益增长的医疗影像设备所产生的具有医学分析和指导价值的结构化和非结构化的海量数据。同时由于标准的数据和规范的标注是医疗人工智能发展的前提,反过来,人工智能可以推动医疗数据的标准化建设。其中结构类影像,比如X光、CT,能够非常直观地观察生理结构,判断是否有物理变化的病变,基于人工智能算法实现图像中解剖结构的准确分类和定位是的全自动诊断的基础。而功能类影像能够研究器官对某种物质的代谢能力,从而反映该器官功能,其缺点是不能自主定位异常,不能直接反映真实生理结构,只能通过影像像素和内容综合理解程度来分析代谢的强弱程度,不能实现具有统计学意义的定量分析,诊断结果只能全凭医生的肉眼和经验来判断,导致较高的误诊漏诊率。若结合人工智能算法在定量、定位、精准量化的基础上,通过与正常数据进行统计比对,大大提高了对病变分析的深度,在实现自动辅助诊断上就具备了现实意义。

其二是有经验的医师缺口大,成长曲线陡峭,医师数量增长远不及影像数据增长,在短时间内理解影像数据给出准确诊断的压力会越来越大,远超负荷。AI算法可以比人类科学家或医学专业人员更快,更准确地提取大量数据,挖掘模式和预测,加强疾病诊断,提供治疗计划。因此从完全由人类设计的系统逐渐转变为由计算机自动提取特征向量进行训练的系统,突破主要来自AI领域的深度学习方法,其模拟更高层次抽象并决定了高维特征空间中的最佳决策边界,对某些疾病的影像诊断水平已能达到专家水准,同时通过对深度学习理论的研究人工智能的决策机制,未来或为实现精准诊疗、智慧医疗和保障大众健康带来突破性进展。这对于提升基层医疗服务水平、助推分级诊疗将具有重大意义。

总而言之,人工智能的核心技术—深度学习,深度学习正在提升所有模式识别的能力,从解剖结构到疾病,让计算机系统能够自主学习经验数据,不仅能更帮助患者更快速地完成健康检查,同时也可以帮助影像医生减少阅片时间,提升效率,降低误诊的概率,既能为经验不足的医生提供辅助决策建议,也能帮助专家节约时间。       

\section{国内外研究现状}

\subsection{图像内容理解的研究现状}

医学影像分析主要涉及图像的内容理解,而实现图像的内容理解是计算机视觉的终极目标\cite{Cootes2004}。计算机视觉的起源可以追溯到1966年,麻省理工AI实验室著名的人工智能学家马文·明斯基给他的本科学生布置暑期项目,让学生构造一个视觉系统。20世纪70年代,David Marr定义了计算机视觉研究的三个层次分成表达、算法、和实现,并提出了一个多层表达,从primal sketch(首要简约图), 到2 ½ D sketch(深度简约图), 到3D sketch,如图\ref{},还包含了纹理、立体视觉、运动分析、表面形状等等\cite{},研究者开始去试图解决让计算机告知他到底看到了什么东西这个问题。研究者希望先把三维结构从图像里面恢复出来,在此基础上再去做理解和判断。
(David Marr,1970s)
20世纪80年代,是人工智能发展的一个非常重要的阶段。人工智能界的逻辑学和知识库推理大行其道,研究者开始做很多专家推理系统,计算机视觉的方法论也开始在这个阶段产生一些改变。人们发现要让计算机理解图像,不一定先要恢复物体的三维结构。例如:让计算机识别一个苹果,假设计算机事先知道苹果的形状或其他特征,并且建立了这样一个先验知识库,那么计算机就可以将这样的先验知识和看到物体表征进行匹配。如果能够匹配,计算机就算识别或者理解了看到的物体。所以,80年代出现了很多方法,包括几何以及代数的方法,将我们已知的物品转化成一些先验表征,然后和计算机看到的物品图像进行匹配。

(David Lowe, 1987)
上世纪90年代,人工智能界又出现了一次比较大的变革,也就是统计方法的出现和流行。在这个阶段,经历了一些比较大的发展点,比如现在还广泛使用的局部特征。前面介绍的形状、颜色、纹理这些表征,其实会受到视角的影响,一个人从不同的角度去看物品,它的形状、颜色、纹理可能都不太一样。随着90年代统计方法的流行,研究者找到了一种统计手段,能够刻画物品最本质的一些局部特征,比如:要识别一辆卡车,通过形状、颜色、纹理,可能并不稳定,如果通过局部特征,即使视角变化了,也会准确对其进行辨识。局部特征的发展,其实也导致了后来很多应用的出现。比如:图像搜索技术真正的实用,也是由于局部特征的出现。研究者可以对物品建立一个局部特征索引,通过局部特征可以找到相似的物品。其实,通过这样一些局部点,可以让匹配更加精准。

(David Lowe, 1999)
到2000年左右,机器学习方法开始盛行。以前需要通过一些规则、知识或者统计模型去识别图像所代表的物品是什么,但是机器学习的方法和以前完全不一样。机器学习能够从海量数据里面去自动归纳物品的特征,然后去识别它。在这样一个时间点,计算机视觉界有几个非常有代表性的工作,比如:人脸识别。
要识别一个人脸,第一步需要从图片里面把待识别的人脸区域给提取出来,这叫人脸检测。像在拍照的时候,会看到相机上有个小方框在闪,那其实是人脸识别必要的第一步工作,也就是人脸框的检测。在以前,这是非常困难的工作,但是在2000年左右,出现了一种非常好的算法,它能够基于机器学习,非常快速的去检测人脸,称之为Viola&Jones人脸检测器,这奠定了当代计算机视觉的一个基础。

(Viola&Jones, 2001)
机器学习的盛行其实是伴随着一个必要条件出现的,就是在2000年左右,整个互联网的出现和爆发,产生了海量的数据,大规模数据集也相伴而生,这为通过机器学习的方法来做计算机视觉提供了很好的土壤。在这期间,出现了大量学术官方的,针对不同领域评测的数据集。人脸检测最具有代表性是一个叫FDDB的数据集。这个数据集包含了超过5000多张人脸数据,每一张人脸,都人为的用框给框出来,机器就可以从这些框好的数据里面,通过一些机器学习的手段去学习从任何一张图片中找到人脸区域。

(FDDB,5171 faces, 2845 images)
另外,我们耳熟能详的一个数据集是人脸识别数据集LFW。提到人脸识别,可能我们听过“计算机做人脸识别比人更准”。其实这句话是有一定问题,在很多实际场景里,计算机做人脸识别连人的70%可能都达不到。那么在什么情况下,它比人准呢。有一种情况就是在LFW上。LFW数据集里面有1万多张人脸和5000多个人的数据,每个人都有在不同情况下拍摄的真实场景的多张人脸。基于这样的一个数据集,研究者可以在上面评估人脸识别的精度。人脸识别领域发展非常快,尤其是最近几年深度学习的兴起后,这个数据集变得非常火爆。人在LFW上面的识别正确率大概是97.75%,而机器已经在LFW上可以做到99.75%甚至更要高。

(13233 faces, 5749 people)
在这期间,还出现了其他非常有影响力的数据集,其中比较有代表性的就是由李飞飞教授发起的一个项目IMAGEMET。他通过众包的方式,标注1400万张图片,分了大概2万多个类别,这些类别包罗万物,比如像动物,里边可能分为鸟类、鱼类等;植物,里面可能会分为树和花。她的理想很宏大,就是希望提供这样一个数据集,为计算机视觉算法提供一个数据土壤,让未来的机器能够认识世界上的一切物品。

在2000年代,人工智能经历了一个快速发展期以后,整个人工智能在很多行业取得了非常好的应用,比如:视觉之外有搜索引擎排序和计算广告等等,视觉领域人脸检测器也被用在了各种各样的相机里面。
到2010年代,人工智能进入了一个最激动人心的年代,它就是深度学习的年代。深度学习从本质上给整个人工智能带来了一次革命。在上世纪,有人做了一个猫的实验。在猫脑袋上开了一个洞,然后在猫的前面放各种各样的物品,观察猫在不同物品下的反应。通过实验发现,当我们放一些形状非常类似的物品时,猫的后脑皮层的相同区域会表现出同样的刺激反应,这一实验了说明人的视觉系统认知是分层的。在早期模拟人脑认知的多层神经网络模型经历2000年左右的一个低谷后,2006年Hinton教授在《科学》发表了对于深层神经网络的训练方法,带来了深度学习的蓬勃发展。

(LeNet-5, 1998)
上图左侧是一个普通的三层神经网络算法,是一种浅层的学习方法。在这种浅层神经网络中,输入的数据依然是人工抽象出来的特征表达,比如颜色、形状、纹理或者局部特征等。而右边的则是现在最常使用的一种深度学习方法,卷积神经网络。从外面来看,最大的改变是其输入是原始图像的像素,不再用人工进行特征表达和抽象,而改用神经网络本身在浅层自动进行特征抽象。这是浅层机器学习模型和深度学习模型的根本区别之一。

(Hinton, 2006)

(Li, Tang etl, 2014-2015)
自2006年开始,在接近10年的时间里,整个计算机视觉界产生了质的变化,深度学习的出现真正改变了计算机视觉之前的定义。


计算机视觉在超声心动图中的应用.心脏回声有一些挑战,医学图像分析可以解决。例如,研究人员建议使用计算机视觉自动分割解剖结构,检测和分类先天性心脏缺陷,实时导管定位等。标准视图采集是心脏超声最基本的任务,也可以通过医学图像分析。标准视图获取 为了找到标准的心脏视图,软件应该从超声波扫描期间的多个帧中选择合适的二维平面。在这里,出现了不同的挑战,如分析二维帧,三维体积,二维时间序列或四维时空图像相关(STIC)体积。因此,随着医学图像数据量的不断增长,我们可以期待医学图像分析软件很快成为超声系统的重要组成部分。
 


\subsection{医学图像分析应用人工智能的研究现状}


人工智能在医学影像的应用主要分为两个部分:第一个部分是图像识别,在前文我们已经作了说明;第二个部分深度学习,是人工智能应用的最核心环节。
人工智能算法通过识别模式来读取医学图像。人工智能系统使用大量检查进行训练,以确定来自CT,磁共振成像(MRI),超声或核成像扫描的正常解剖结构。然后使用异常情况训练AI系统的眼睛以识别异常,类似于计算机辅助检测软件(CAD)。然而,与CAD只是放射科医生可能想要仔细研究的区域不同,AI软件具有更多的分析认知能力,基于更多的前几代CAD软件的临床数据和阅读体验。
2006年,神经网络领域的大师Geoffrey Hinton教授与其博士生在《Science》和相关期刊上发表了论文,首次提出了“深度信念网络”的概念。与传统的训练方式不同,“深度信念网络”有一个“ 预训练”(pre-training)的过程,这可以方便的让神经网络中的权值找到一个接近最优解的值,之后再使用“ 微调 ”(fine-tuning)技术来对整个网络进行优化训练。这两个技术的运用大幅度减少了训练多层神经网络的时间。他给多层神经网络相关的学习方法赋予了一个新名词–“ 深度学习 ”。

2012年,Hinton教授的研究团队参加了斯坦福大学Fei-Fei Li教授等组织的ImageNet ILSVRC大规模图像识别评测任务。该任务包括120万张高分辨率图片,1000个类比。Hinton教授团队使用了全新的黑科技多层卷积神经网络结构,突破性地将图像识别错误率从26.2%降低到了15.3%。 这一革命性的技术,让神经网络深度学习以极快的速度跃入了医疗和工业领域,这才有了后来一系列使用该技术的医学影像公司的出现。

比如国际知名的医学影像公司Enlitic和国内刚刚获得有峰瑞资本600万天使轮融资的DeepCare。都是通过积累大量影像数据和诊断数据,来不断对的神经元网络进行深度学习训练,从而提高医生诊断的准确率。

以Enlitic公司开发的恶性肿瘤检测系统为例,它通过使用肺癌相关图像数据库 “LIDC(Lung Image Database Consortium)”和“NLST(National Lung Screening Trial)”进行验证,结果发现,该公司开发的系统的肺癌检出精度比一名放射技师检查肺癌的精度高5成以上。

机器学习用于医学成像领域,包括计算机辅助诊断,图像分割,图像配准,图像融合,图像引导治疗,图像标注和图像数据库检索。这意味着医学领域有很多领域,可以应用机器学习方法,并可以帮助改善患者的医疗保健[77]。例如,机器学习方法可以用于乳腺癌的早期检测[78]。机器学习方法可解决的一个具体任务是分类,根据输入特征对对象进行分类(例如,异常或正常,良性或恶性)[13,74,76]。计算机断层扫描的肺部疾病分类是机器学习方法应用于此任务的一个例子[69,74]。
深度学习方法是机器学习中的一组算法,可以学习多种算法代表性和抽象级别有助于理解数据[7]。高级抽象是从低级抽象定义的,因此可以学习更复杂的函数。特别是,如果一个函数可以用深层体系结构紧凑地表示,如果这个体系结构的深度变得更浅,相同的函数可能需要一个非常大的体系结构[7]。
“当前影像诊断主要依赖人工阅片完成,然而,日益增加的图像数据也为人工阅片带来极大挑战。为了给医生提供有效的辅助诊断信息,智能图像处理技术正变得越来越重要。以机器学习和图像处理技术为基础的计算机辅助诊断(computeraided diagnosis,CAD)逐渐成为医学领域的研究热点[1]。基于机器学习的 CAD 主要包括四方面的内容:① 图像预处理;② 感兴趣区( r e g i o n of interest,ROI)的分割;③ 特征提取、选择与分类;④ 肿瘤区域的识别(分类或者分割)[5]。其中,高效特征的提取尤为关键[6]。目前,基于传统的浅层机器学习结构的 CAD 系统,高度依赖人工选择的特征,以及分类器对特征的整合。而且,由于传统的浅层学习结构无法满足实际应用中对复杂函数建模的要求[7],所以难以区分高维特征之间的关系,通常需要降维处理。因此,我们需要简化及优化 CAD 技术中的特征选择的过程,以提高 CAD 系统进行辅助诊断的准确度。 

深度学习是2013年的十大突破性技术之一[2],并且在过去几年中获得了爆炸性增长;见图1.图2概括了一幅总结机器学习方法与医学成像问题之间协同作用的大图,该图形定义了本文提倡的新兴深度成像领域在过去的二十年里,注意机器学习呈指数增长,比医学成像有更强的趋势。 交叉点(红色)表示近期在医学成像和深度学习方面的研究成果相当(五十五)。
作为神经系统的中心,人脑中含有数十亿个神经元[3]。神经科学将大脑视为具有负责人类智能的复杂生物神经网络的生物“计算机”[3]。在工程师看来,神经元是一个电信号处理单元。一旦神经元被激发,离子泵就会在膜两侧保持电压,通过膜中的离子通道产生离子浓度差异。如果电压充分改变,动作电位被触发沿着轴突通过突触连接到另一个神经元。整个神经网络的动态还远远没有被完全理解。受生物神经网络的启发,人造神经元被设计为人工神经网络(ANN)的元素[4]。该模型将输入端口处的数据线性组合到树状结构中,并非线性地将加权和转换为输出端口,如轴突。
虽然人工神经网络方法的动机很好,但大约二十年来,这种和其他机器学习技术并没有引起公众的兴奋,直到几年前深度学习成为流行语。对神经网络的批评之一就是需要大量的数据,训练时间随网络规模和问题复杂性缩小,以及模型识别可能陷入局部极值的风险。在去年剑桥大学的一次演讲中,多伦多大学的杰弗里•辛顿博士解释了深层神经网络如何取得了令人振奋的突破。简而言之,启用因素是多重的:数千倍的数据(大数据),数百万倍的计算能力(GPU,TPU等),更智能的权重初始化(分层,传输等),更好非线性变换(ReLU等)以及更深的网络拓扑。作为一个非凡的里程碑[5],一个受限的玻尔兹曼机器(RBM)的无监督学习过程可以被有效地递归地用于在没有监督的情况下逐层准备深层网络。然后,预先训练的参数可以通过反向传播进行微调。深度网络的成功现在在计算机视觉,语音识别,语言处理,经典和电子游戏等领域得到了很好的报道,最近的高光“AlphaGo”(这个计算机程序扮演棋盘棋和首次击败职业球员)[6]。

我们不是要覆盖很多机器学习的技术细节,而是先看一个模式识别任务的神经网络,例如人脸识别。如图3所示,深层网络中有许多层间连接的神经元。数据被馈送到网络的输入层。与神经元相关联的权重通常在预训练和微调过程或具有大量未标记和标记图像的混合训练过程中获得。结果在网络的输出层中获得。其他图层隐藏直接访问。每个图层都使用前一个图层的特征来形成更高级的特征。在较早的图层中,会分析较低级别的特征,例如边缘,角落和面部图案。在后面的图层中,更高级的特征被合成以匹配脸部模板。由于过去几年已经开发出了创新的算法成分,这种深度学习机制在文献[7] - [9]中报道的图像特征提取方面取得了极大的成功。请注意,深层网络与许多其他多分辨率分析方案和优化方法有根本的不同。深度网络的一个特点是对于机器智能过于复杂的巨大维度问题的非线性学习和非凸优化能力。 

“近年来方兴未艾的深度学习技术[8] 作为一类多层神经网络学习算法,可通过深层非线性网络结构学习特征,并且通过组合低层特征形成更加抽象的深层表示(属性类别或特征),实现复杂函数逼近,表征输入数据分布式表示,从而可以学习到数据集的本质特征[7]。因此,深度学习算法应用于 CAD 系统具有以下优势:第一,作为一种数据驱动的自动特征学习算法,可以直接从训练数据提取特征,从而大大减少特征提取的工作量以及人工干预的影响;第二,通过神经网络内在的深层结构可以表征特征之间的交互及层次结构,从而揭示高维特征之间的联系;第三,特征提取、特征选择及特征分类。三个核心步骤可以在同一个深层结构的最优化中实现[6]。由此可见,深度学习有望解决基于传统浅层机器学习的 CAD 问题,从而大大提高辅助诊断能力。
医学图像分析是计算机视觉的实际应用-计算机科学的一个分支,涉及数字图像(包括数字视频帧)中的对象和特征识别。计算机视觉算法通过一系列过程来分析图像,类似于人类视觉系统所执行的过程。在经过初步预处理(包括去噪,滤波和特征增强)之后,软件在图像分割的过程中将图像分解成有意义的区域。然后,算法提取重要的特征,并基于这些特征对图像中的对象进行分类。此外,医学图像分析算法通常执行图像配准 - 映射两个以上相同解剖结构的图像以检测任何差异或变化。
基于机器学习,分类是医学图像分析软件最复杂的功能。每个AI系统都使用机器学习方法作为其“大脑”。这些算法允许计算机记住大量信息,并在学习完成后使用它来分析类似的信息。这就是为什么这种方法在计算机视觉中得到如此广泛的应用在图像数据集(例如超声图像数据集)上进行训练,然后软件识别真实世界图像中的熟悉特征(例如,在实时超声扫描中)和在此基础上作出相关的结论。这些系统的准确性随着输入数据的数量而增加。从数百个图像开始,它们显示出不错的结果,并且在处理了数以千计的图像和更多图像之后,它们的准确度接近100%。当然,这也取决于所使用的架构,随着机器学习方法的发展,用于医学图像分析的算法显示出更好的结果。

西门子医疗集团率先将人工智能(AI)算法引入心脏回波系统,以加速自动化。几年前,飞利浦医疗保健公司也在其Epiq超声系统中引入了AI的元素。它需要一个三维回波数据集采集和自动分析图像,以确定心脏的解剖,标签,然后切片的最佳标准视图呈现。这消除了互操作性差异的问题,因为软件将总是选择基于机器学习的最佳视图,该机器学习使用数千个代表患者解剖变异谱的先前检查。这对于操作人员来说要积累相同的知识需要花费数年的时间。其他供应商也引入了深度学习算法的元素来帮助分析超声心动图或执行自动量化。下一代回声系统将结合更多的人工智能功能,通过自动完成耗时的任务和扩大超声检查员的工作量,从而进一步改善工作流程,从而提高工作效率,始终保持准确。

医学影像技术在我国医疗系统中的发展时间比较短,所以在技术方面还不够成熟,但是随着医疗技术以及影像技术的不断发展,首先,医学影像技术呈现出来的信息必然会更加具有敏感性、直观性以及特异性;其次,现在对影像的分析都是定性分析,在未来必然会向着定量的方向发展,不再仅仅给出疾病的诊断结果,而是向着提供手术路径的方向发展;再次,影像信息的采集与显示都还是二维图像,随着数字成像技术的不断发展必然会向着三维全数字化发展;最后,目前,放射科在使用影像技术进行疾病诊断的过程中使用的都还是单一技术,随着影像技术的不断进步,未来会逐渐引进新的影像技术,向着综合方向发展。 通用电气,西门子和飞利浦是超声心动图供应商之一,将深度学习算法整合到回声软件中,帮助自动从三维超声数据集提取标准成像视图。这是飞利浦Epiq系统的一个例子,该系统使用供应商的解剖智能软件来定义解剖结构,并自动显示解剖标准诊断视图,无需人工干预。这可以大大加快工作流程并减少操作员之间的差异。

包括几家分析公司和创业公司在内的其他公司则展示了使用AI快速筛选大量大数据的软件,或者为适当的使用标准提供即时的临床决策支持,最好的测试或成像来进行诊断甚至提供差异诊断。飞利浦将AI作为其具有自适应智能的新型Illumeo软件的一个组件,该软件可自动获取相关的放射科先前的检查结果。用户可以在特定的MPI视图中点击解剖结构的区域,AI将查找并打开先前的成像研究以显示相同的解剖结构,切片和方向。对于肿瘤学成像,在图像中点击几次肿瘤,AI将执行自动量化,然后对先验进行相同的测量,呈现肿瘤评估的并排比较。这可以显着减少与肿瘤跟踪评估和加速工作流程相关的时间。

基于人工智能(AI)的医学图像分析采用卷积神经网络,支持向量机,模糊逻辑系统等机器学习方法从医学图像中提取意义。最先进的计算机视觉软件为诊断人员提供了基于证据的技巧,消除了可能的疑惑并确保了诊断的一致性。标准视图位置是超声心动图中的关键步骤,因为这些帧包含基本的诊断数据。从超声波检查自动捕捉标准飞机可以加快扫描,并使其更加准确。仔细研究这方面的研究将证明这不是一个猜测。标准视图的计算机辅助检测不断支持临床医生。

\section{创新点及全文结构}

所在的研究组多年来与四川大学华西医院合作展开医学影像处理系统中关键技术的研究。博士期间在相关课题资助下,通过分析,抓住其中的关键问题,即影响手术精确度的术前诊断及规划中的肿瘤分割与手术进行中的肿瘤配准问题,并对上述问题进行研究及解决算法的改进。

主要研究内容及成果如下:

1)通过构建标准切面数据库,提出了一种基于深度卷积神经网络的超声心动图标准切面自动识别方法,该算法针对网络全连接层占有模型大部分参数的缺点,引入空间金字塔均值池化替代全连接层,获得更多的空间结构信息,利用全局空间金字塔均值池化方法进行微调迁移学习,并大大减少模型参数、降低过拟合风险,通过类别显著性区域将类似注意力机制引入模型可视化过程,详尽分析了数据规模对模型分类精度的影响,并对模型的可解释性和有效性进行了分析。

2)针对基于深度卷积神经网络的图像分类模型的可解释性问题,通过评估模型特征空间的潜在可表示性,提出一种用于改善理解模型特征空间的可视化方法。给定任何已训练的深度卷积网络模型,引入了通过激活最大化获得的图像可解释性的正则化方法,结合现有正则化方法提出空间金字塔分解方法,利用构建多层拉普拉斯金字塔主动提升目标图像特征空间的低频分量,结合多层高斯金字塔调整其特征空间的高频分量得到较优可视化效果。并通过限制可视化区域,提出利用类别显著性激活图技术加以压制上下文无关信息,可进一步改善可视化效果。该模型有效克服了原有可视化方法中由于不能主动调整高低频分量等原因造成的可视化图像语义重复和低效率等问题。

3)针对自动检测医学图像中指定目标时存在的问题,提出了一种基于深度学习自动检测目标位置和估计对象姿态的算法。该算法基于区域深度卷积神经网络和目标结构的先验知识,采用区域生成候选框网络、感兴趣区域池化策略,引入包括分类损失、边框位置回归定位损失和像平面内朝向损失的多任务损失函数,近似优化一个端到端的有监督定位网络,能快速地对医学图像中目标自动定位,有效地为下一步的分割和参数自动提取提供定位结果。并在超声心动图左心室检测中提出利用检测额外标记点:二尖瓣环、心内膜垫和心尖,能高效地对左心室朝向姿态进行估计。

4)针对图像去噪中存在的问题,我们提出了一个有监督多层残差卷积网络框架,结合不同损失函数学习端到端映射变换。输入是带噪声的图像和原图像,输出的是去噪后的图像。基于传统卷积神经网络的心室图像分割方法研究,传统手工分割方法费时、精确度低,易受操作者经验影响,而传统机器学习算法需要手动筛选设计特征,普适性低,因此本文提出能够综合运用多模图像信息且自动提取结构性特征的卷积神经网络方法。本文设计了不同架构的单通道CNNs分割模型,与如今流行的分割方法比较,大大提高了分割与识别正确率。 

\section{论文的章节安排}

全篇共五章,结构如下:

第一章绪论介绍了应用人工智能进行医学影像分析的研究背景及意义,对当前研究现状及难点进行剖析,同时阐述了本论文的研究内容,列出了主要创新点,最后给出了整篇文章的章节结构。

第二章描述了本文基于传统卷积神经网络的肿瘤分割算法。在这章中,首先分析了传统肿瘤分割算法及学习类算法存在的弊端;接着介绍了传统的医学图像分割算法;然后分析了传统分割类方法存在的问题与不足,引出单通道CNNs模型并将其应用到肿瘤图像分割中;最后给出基于CNNs的肿瘤分割与识别架构,
同时测试了不同实验数据下算法的性能并与当前流行算法进行了比较。

第三章描述了本文的基于多通道CNNs的多体位肿瘤分割方法。该章首先描述了传统CNNs算法中存在的不足;然后提出了一种综合局部及全局信息的多通道卷积神经网络模型用于精确分割;最后给出该算法的整体流程图;实验结果表明本章方法优于传统CNNs模型,且配准精度能与目前最流行的肿瘤分割算法相匹配。

第四章介绍了本文提出的深度迭代Log-demons的配准算法。首先描述了传统配准框架在存在大的变形时失效的问题;然后介绍了医学图像配准中的基本概念;接着提出了深度迭代配准框架;最后提出了基于CNNs的图像预配准技术,并将其揉合到本文提出的双层迭代配准框架下,实验结果验证了本文方法的鲁棒性。 

第五章分析了本文提出的基于PCA的相似性测度算法。首先,分析了本章算法提出的问题背景,即要解决传统配准中运算速度慢的问题;之后列出了类似的相似性测度原理及方法;然后详细介绍了基于PCA的相似性测度优化,即充分利用图像中的主要特征并结合传统测度进行优化;最后从三维及二维数据全面验证本文方法的有效性和鲁棒性。

最后结论部分总结全文,并展望了今后的研究工作。