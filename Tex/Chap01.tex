\chapter{绪论}
\label{chap:introduction}

\section{研究背景及现实意义}
\subsection{医学影像分析的研究背景}

临床医学历经几百年的发展,传统的“视、触、叩、听”已经不能满足现代化医疗的诊断需求,医学影像极大地变革了传统的诊疗体系,成熟的成像模式不断完善,新技术不断涌现,在疾病筛查、早期诊断、治疗方案选择和预后评估等方面发挥着举足轻重的作用。用于医疗诊断的影像使医生能够更早地发现疾病并改善患者预后,介入或术中成像有助于消除和治愈许多检测到的疾病,能更早更有效地诊断身心健康状况,为临床诊疗提供了全面的视角和丰富的信息,迅速地被广泛应用于临床领域。目前临床医学已经无法离开医学影像,并且随着医学影像的发展,临床诊疗将越来越依赖于影像。

医学影像自1895年伦琴发现X射线以来,综合利用物理中的各种物质波、光电子技术以及计算机技术,从宏观到微观,由静态到动态,由单模到多模,由2D到3D,形成了各种的成像模式,包括X射线(X-ray),超声(Ultrasound,US),计算机断层扫描(Computed Tomography,CT),磁共振成像(Magnetic Resonance Imaging,MRI),正电子断层扫描(Positron Emission Computed Tomography , PET ),单光子断层扫描( Single-Photon  Emission Computed Tomography,SPECT),以及内窥镜,病理切片等。

医学影像分析也从传统的计算机辅助检测发展到火热的影像组学,但世界医疗卫生系统每天都会浪费大量的资源和时间,对医学影像内容的错误理解会造成错误诊断,导致很多不必要的额外检查,导致治疗计划的延迟,大大减少了如果早期正确发现的生存率或缓解率。同时不同的影像质量和不同的工作流程会导致临床上医生对影像内容的理解具有很大的主观性,对于通常在临床筛查和风险评估时所获得的医学图像,这些结构可能具有相当模糊的边界和低对比度,即使对于有经验的临床医生来说,图像级解释也是具有挑战性和耗时的任务。

“当前影像诊断主要依赖人工阅片完成,然而,日益增加的图像数据也为人工阅片带来极大挑战。为了给医生提供有效的辅助诊断信息,智能图像处理技术正变得越来越重要。以机器学习和图像处理技术为基础的计算机辅助诊断(computeraided diagnosis,CAD)逐渐成为医学领域的研究热点[1]。基于机器学习的 CAD 主要包括四方面的内容:① 图像预处理;② 感兴趣区( r e g i o n of interest,ROI)的分割;③ 特征提取、选择与分类;④ 肿瘤区域的识别(分类或者分割)[5]。其中,高效特征的提取尤为关键[6]。目前,基于传统的浅层机器学习结构的 CAD 系统,高度依赖人工选择的特征,以及分类器对特征的整合。而且,由于传统的浅层学习结构无法满足实际应用中对复杂函数建模的要求[7],所以难以区分高维特征之间的关系,通常需要降维处理。因此,我们需要简化及优化 CAD 技术中的特征选择的过程,以提高 CAD 系统进行辅助诊断的准确度。

“近年来方兴未艾的深度学习技术[8] 作为一类多层神经网络学习算法,可通过深层非线性网络结构学习特征,并且通过组合低层特征形成更加抽象的深层表示(属性类别或特征),实现复杂函数逼近,表征输入数据分布式表示,从而可以学习到数据集的本质特征[7]。因此,深度学习算法应用于 CAD 系统具有以下优势:第一,作为一种数据驱动的自动特征学习算法,可以直接从训练数据提取特征,从而大大减少特征提取的工作量以及人工干预的影响;第二,通过神经网络内在的深层结构可以表征特征之间的交互及层次结构,从而揭示高维特征之间的联系;第三,特征提取、特征选择及特征分类。三个核心步骤可以在同一个深层结构的最优化中实现[6]。由此可见,深度学习有望解决基于传统浅层机器学习的 CAD 问题,从而大大提高辅助诊断能力。

近来,结合机器学习和计算机视觉的人工智能算法被用来帮助临床医生,提高医师对患者影像的理解,从而改善诊断,治疗以及由此产生的预后效果,将作为一个固化已有经验的临床助理,给临床医生的工作方式带来转变,显著提高工作流程效率,而不增加临床医生的负担。机器学习已被用于医学图像分析,计算机及其运行的算法可以比人类科学家或医学专业人员更快,更准确地提取大量数据,挖掘模式和预测,加强疾病诊断,提供治疗计划。机器学习通常始于机器学习算法系统,该系统计算被认为在进行感兴趣的预测或诊断中是重要的图像特征。然后,机器学习算法系统识别这些图像特征的最佳组合,以对图像进行分类或计算给定图像区域的一些度量。图像中解剖结构的准确分类和定位是的基于图像全自动诊断的基础。

\subsection{课题研究意义}

人工智能进入医学成像领域,数据科学革命大约在五年前随着IBM Watson和Google Brain的出现而开始。他说,2012年推出的深度学习算法确实推动了人工智能的发展,到2014年,机器正确读取放射学研究的比例开始下降,准确度达到了95%左右。回顾旧的检查,以帮助医院找到病人可能没有意识到病情的新病人。通过在卫生系统中记录的所有先前的胸部CT检查来帮助识别可能患有肺癌的患者。总的来说,人工智能提供了一个重要的机会来增强和增强放射科的阅读,而不是取代放射科医生。医疗领域的人工智能和机器学习将继续得到改善,影响疾病预防和诊断,

西门子医疗集团率先将人工智能(AI)算法引入心脏回波系统,以加速自动化。几年前,飞利浦医疗保健公司也在其Epiq超声系统中引入了AI的元素。它需要一个三维回波数据集采集和自动分析图像,以确定心脏的解剖,标签,然后切片的最佳标准视图呈现。这消除了互操作性差异的问题,因为软件将总是选择基于机器学习的最佳视图,该机器学习使用数千个代表患者解剖变异谱的先前检查。这对于操作人员来说要积累相同的知识需要花费数年的时间。其他供应商也引入了深度学习算法的元素来帮助分析超声心动图或执行自动量化。下一代回声系统将结合更多的人工智能功能,通过自动完成耗时的任务和扩大超声检查员的工作量,从而进一步改善工作流程,从而提高工作效率,始终保持准确。

人工智能算法通过识别模式来读取医学图像。人工智能系统使用大量检查进行训练,以确定来自CT,磁共振成像(MRI),超声或核成像扫描的正常解剖结构。然后使用异常情况训练AI系统的眼睛以识别异常,类似于计算机辅助检测软件(CAD)。然而,与CAD只是放射科医生可能想要仔细研究的区域不同,AI软件具有更多的分析认知能力,基于更多的前几代CAD软件的临床数据和阅读体验。出于这个原因,正在帮助开发医学人工智能的专家经常将认知能力称为“有效的CAD”。

通过学术界及企业界的共同努力,我们获得了大量的各种各样的医学影像数据,并形成了多个面向全球公开的医学影像数据库。 大量医学影像数据的不断采集与积累,为手术导航精确制导提供了契机,但如何充分有效地利用这些影像,也给临床医生及医学影像信息工作者的研究提出了巨大的挑战。处理此如巨大数量的医学影像数据,亟待解决的问题便是如何综合利用不同模式间影像的互补信息,以及消除不同影像维度及分辨率带来的影响。分析和处理这些大数据,从中准确挖掘有用信息,需要我们提出更快速、鲁棒的计算方法。越来越进步的计算机辅助技术,如模式识别技术、数据挖掘技术等与大规模图像识别和机器学习技术结合,为大规模的数据分析和处理提供了可能。传统手工勾勒分割及配准方法工作量大,耗时多,且受医生自身经验的影响。而传统分割与配准算法缺乏自主学习性,需要算法设计者手动设计特征,且适用于单一种类图像,当学习对象改变时,需要重新进行训练。最近计算机视觉领域流行的深度学习类算法,对不同模式及维度的图像进行训练,使得算法具备自主学习性及普适性,便于更加精确地辅助肿瘤手术,因此为临床提供了新的诊疗契机。       

\section{国内外研究现状}

\subsection{医学图像分析应用计算机视觉的研究现状}

计算机视觉在超声心动图中的应用.心脏回声有一些挑战,医学图像分析可以解决。例如,研究人员建议使用计算机视觉自动分割解剖结构,检测和分类先天性心脏缺陷,实时导管定位等。标准视图采集是心脏超声最基本的任务,也可以通过医学图像分析。标准视图获取 为了找到标准的心脏视图,软件应该从超声波扫描期间的多个帧中选择合适的二维平面。在这里,出现了不同的挑战,如分析二维帧,三维体积,二维时间序列或四维时空图像相关(STIC)体积。因此,随着医学图像数据量的不断增长,我们可以期待医学图像分析软件很快成为超声系统的重要组成部分。
 
医学图像分析是计算机视觉的实际应用-计算机科学的一个分支,涉及数字图像(包括数字视频帧)中的对象和特征识别。计算机视觉算法通过一系列过程来分析图像,类似于人类视觉系统所执行的过程。在经过初步预处理(包括去噪,滤波和特征增强)之后,软件在图像分割的过程中将图像分解成有意义的区域。然后,算法提取重要的特征,并基于这些特征对图像中的对象进行分类。此外,医学图像分析算法通常执行图像配准 - 映射两个以上相同解剖结构的图像以检测任何差异或变化。
基于机器学习,分类是医学图像分析软件最复杂的功能。每个AI系统都使用机器学习方法作为其“大脑”。这些算法允许计算机记住大量信息,并在学习完成后使用它来分析类似的信息。这就是为什么这种方法在计算机视觉中得到如此广泛的应用在图像数据集(例如超声图像数据集)上进行训练,然后软件识别真实世界图像中的熟悉特征(例如,在实时超声扫描中)和在此基础上作出相关的结论。这些系统的准确性随着输入数据的数量而增加。从数百个图像开始,它们显示出不错的结果,并且在处理了数以千计的图像和更多图像之后,它们的准确度接近100%。当然,这也取决于所使用的架构,随着机器学习方法的发展,用于医学图像分析的算法显示出更好的结果。

\subsection{医学图像分析应用人工智能的研究现状}

医学影像技术在我国医疗系统中的发展时间比较短,所以在技术方面还不够成熟,但是随着医疗技术以及影像技术的不断发展,首先,医学影像技术呈现出来的信息必然会更加具有敏感性、直观性以及特异性;其次,现在对影像的分析都是定性分析,在未来必然会向着定量的方向发展,不再仅仅给出疾病的诊断结果,而是向着提供手术路径的方向发展;再次,影像信息的采集与显示都还是二维图像,随着数字成像技术的不断发展必然会向着三维全数字化发展;最后,目前,放射科在使用影像技术进行疾病诊断的过程中使用的都还是单一技术,随着影像技术的不断进步,未来会逐渐引进新的影像技术,向着综合方向发展。 通用电气,西门子和飞利浦是超声心动图供应商之一,将深度学习算法整合到回声软件中,帮助自动从三维超声数据集提取标准成像视图。这是飞利浦Epiq系统的一个例子,该系统使用供应商的解剖智能软件来定义解剖结构,并自动显示解剖标准诊断视图,无需人工干预。这可以大大加快工作流程并减少操作员之间的差异。

包括几家分析公司和创业公司在内的其他公司则展示了使用AI快速筛选大量大数据的软件,或者为适当的使用标准提供即时的临床决策支持,最好的测试或成像来进行诊断甚至提供差异诊断。飞利浦将AI作为其具有自适应智能的新型Illumeo软件的一个组件,该软件可自动获取相关的放射科先前的检查结果。用户可以在特定的MPI视图中点击解剖结构的区域,AI将查找并打开先前的成像研究以显示相同的解剖结构,切片和方向。对于肿瘤学成像,在图像中点击几次肿瘤,AI将执行自动量化,然后对先验进行相同的测量,呈现肿瘤评估的并排比较。这可以显着减少与肿瘤跟踪评估和加速工作流程相关的时间。

基于人工智能(AI)的医学图像分析采用卷积神经网络,支持向量机,模糊逻辑系统等机器学习方法从医学图像中提取意义。最先进的计算机视觉软件为诊断人员提供了基于证据的技巧,消除了可能的疑惑并确保了诊断的一致性。标准视图位置是超声心动图中的关键步骤,因为这些帧包含基本的诊断数据。从超声波检查自动捕捉标准飞机可以加快扫描,并使其更加准确。仔细研究这方面的研究将证明这不是一个猜测。标准视图的计算机辅助检测不断支持临床医生。

\section{创新点及全文结构}

所在的研究组多年来与四川大学华西医院合作展开医学影像处理系统中关键技术的研究。博士期间在相关课题资助下,通过分析,抓住其中的关键问题,即影响手术精确度的术前诊断及规划中的肿瘤分割与手术进行中的肿瘤配准问题,并对上述问题进行研究及解决算法的改进。

主要研究内容及成果如下:

1)通过构建标准切面数据库,提出了一种基于深度卷积神经网络的超声心动图标准切面自动识别方法,该算法针对网络全连接层占有模型大部分参数的缺点,引入空间金字塔均值池化替代全连接层,获得更多的空间结构信息,利用全局空间金字塔均值池化方法进行微调迁移学习,并大大减少模型参数、降低过拟合风险,通过类别显著性区域将类似注意力机制引入模型可视化过程,详尽分析了数据规模对模型分类精度的影响,并对模型的可解释性和有效性进行了分析。

2)针对基于深度卷积神经网络的图像分类模型的可解释性问题,通过评估模型特征空间的潜在可表示性,提出一种用于改善理解模型特征空间的可视化方法。给定任何已训练的深度卷积网络模型,引入了通过激活最大化获得的图像可解释性的正则化方法,结合现有正则化方法提出空间金字塔分解方法,利用构建多层拉普拉斯金字塔主动提升目标图像特征空间的低频分量,结合多层高斯金字塔调整其特征空间的高频分量得到较优可视化效果。并通过限制可视化区域,提出利用类别显著性激活图技术加以压制上下文无关信息,可进一步改善可视化效果。该模型有效克服了原有可视化方法中由于不能主动调整高低频分量等原因造成的可视化图像语义重复和低效率等问题。

3)针对自动检测医学图像中指定目标时存在的问题,提出了一种基于深度学习自动检测目标位置和估计对象姿态的算法。该算法基于区域深度卷积神经网络和目标结构的先验知识,采用区域生成候选框网络、感兴趣区域池化策略,引入包括分类损失、边框位置回归定位损失和像平面内朝向损失的多任务损失函数,近似优化一个端到端的有监督定位网络,能快速地对医学图像中目标自动定位,有效地为下一步的分割和参数自动提取提供定位结果。并在超声心动图左心室检测中提出利用检测额外标记点:二尖瓣环、心内膜垫和心尖,能高效地对左心室朝向姿态进行估计。

4)针对图像去噪中存在的问题,我们提出了一个有监督多层残差卷积网络框架,结合不同损失函数学习端到端映射变换。输入是带噪声的图像和原图像,输出的是去噪后的图像。基于传统卷积神经网络的心室图像分割方法研究,传统手工分割方法费时、精确度低,易受操作者经验影响,而传统机器学习算法需要手动筛选设计特征,普适性低,因此本文提出能够综合运用多模图像信息且自动提取结构性特征的卷积神经网络方法。本文设计了不同架构的单通道CNNs分割模型,与如今流行的分割方法比较,大大提高了分割与识别正确率。 

\section{论文的章节安排}

全篇共五章,结构如下:

第一章绪论介绍了应用人工智能进行医学影像分析的研究背景及意义,对当前研究现状及难点进行剖析,同时阐述了本论文的研究内容,列出了主要创新点,最后给出了整篇文章的章节结构。

第二章描述了本文基于传统卷积神经网络的肿瘤分割算法。在这章中,首先分析了传统肿瘤分割算法及学习类算法存在的弊端;接着介绍了传统的医学图像分割算法;然后分析了传统分割类方法存在的问题与不足,引出单通道CNNs模型并将其应用到肿瘤图像分割中;最后给出基于CNNs的肿瘤分割与识别架构,
同时测试了不同实验数据下算法的性能并与当前流行算法进行了比较。

第三章描述了本文的基于多通道CNNs的多体位肿瘤分割方法。该章首先描述了传统CNNs算法中存在的不足;然后提出了一种综合局部及全局信息的多通道卷积神经网络模型用于精确分割;最后给出该算法的整体流程图;实验结果表明本章方法优于传统CNNs模型,且配准精度能与目前最流行的肿瘤分割算法相匹配。

第四章介绍了本文提出的深度迭代Log-demons的配准算法。首先描述了传统配准框架在存在大的变形时失效的问题;然后介绍了医学图像配准中的基本概念;接着提出了深度迭代配准框架;最后提出了基于CNNs的图像预配准技术,并将其揉合到本文提出的双层迭代配准框架下,实验结果验证了本文方法的鲁棒性。 

第五章分析了本文提出的基于PCA的相似性测度算法。首先,分析了本章算法提出的问题背景,即要解决传统配准中运算速度慢的问题;之后列出了类似的相似性测度原理及方法;然后详细介绍了基于PCA的相似性测度优化,即充分利用图像中的主要特征并结合传统测度进行优化;最后从三维及二维数据全面验证本文方法的有效性和鲁棒性。

最后结论部分总结全文,并展望了今后的研究工作。