\chapter{相关知识}
\label{chap:basicknowledge}
 
\section{深度卷积神经网络的组件}

 
\subsection{网络结构}

深度CNN是多层前馈神经网络的一种特例。隐藏层的神经元设计成跟上一层神经元局部连接,并利用参数共享来减少模型复杂度。针对图像这种结构化数据,由不同卷积核来探测不同空间位置上的局部统计特征。通过堆叠多层的卷积结构,实现从低层到高层语义空间的抽象映射。

深度CNN的典型结构是在LeNet模型\citep{Jarrett2009}的基础上引入修正线性单元(Rectified Linear Units,ReLU)的激活函数和Dropout等技术\citep{Krizhevsky2012}进行了改进。\ 为CNN模型的网络结构示意图。定义图像数据为 ,且其类别标签 ,其中 和 ,k为类别数, 作为网络输入,输入层的 ,即原始图像作为输入,第 层输出 个大小为 的特征图。第一层为由  个特征图作为输入的卷积层,特征图大小为  。第 层第 特征图定义为  。计算公式为:
      (1)
其中 为偏置矩阵, 为连接第 层第 个特征图和第 层第 个特征的卷积核。
模型的激活函数没有采用Sigmoid函数或双曲正切函数,而是选择ReLU函数,目的是引入更多非线性来加速训练收敛速度,解决多层网络反向传播中梯度弥散的问题。其函数表达式为: 
             
 其中 表示对第 层的激活函数,该层一般嵌入在卷积层后。为了使得每层输入的分布更平稳,一般引入批量归一化层(Batch Normalization, BN),如图1中所示。最大池化层进行下采样,有时把“卷积-激活-归一化-池化”统称为卷积层。最后需连接全连接层(图中Fc层表示),全连接层就不再保存空间信息,是对低层特征的高层抽象,最终输出K维的向量,作为该图像的特征向量送入最终的分类器进行分类评估。

\subsection{反向传播算法} 
图 1 卷积网络模型结构示意图
Fig.1 The structure of convolutions model
深度CNN模型的分类器与传统方法不同的是:把特征提取过程中的卷积核参数和分类器的参数整合到端到端的模型中。对一个有监督的多分类问题,特征提取过程可表示为得分函数 ,W,b是各层可学习的参数包括卷积核K,偏置B和全连接层的权值参数。对第 个样本的得分函数分类误差的交叉熵损失函数可定义为:
  	 
通过最小化Softmax函数的非负对数似然(公式5),能带来归一化的概率解释。一般采用L2损失正则化技术提升分类泛化性能。全部N个样本的损失函数L为公式6所示。其中  表示正则化参数。模型最小化方法采用反向传播算法,通过带动量的批随机梯度下降算法不断调整参数使得模型整体误差函数不断降低。并通过使用权重衰减项和Dropout技术控制过拟合。具体实现详情请参考文献[10]。

\section{AAM模型和CLM模型} 
\subsection{基于AAM的分割} 
AAM是常用医学图像分割方法之一,是用来解释特定对象形状和外观视觉变化的参数生成模型。设m个图像内标记点集合的坐标  ,则第 个形状向量可定义为: 。 AAM对新图像进行分割时,拟合策略通常被构造为最佳形状p和纹理c参数的正则化搜索过程。最小化参数同时依赖于所有标记点位置全局测量偏差:
  (1)
式中R是惩罚形状和纹理变形的正则化项,D是量化给定全局测量偏差的数据项。 和 为对角矩阵包含与形状和纹理特征向量相关联的特征值, 是图像噪声估计。原始匹配算法使用的是线性回归方法[6]。
可以通过假设以下的形状和纹理的概率生成模型来获得式1的概率解释[16,17]:  
     (2)
	  	(3)
式2,3为形状模型和外观模型的概率解释,其假设  服从零均值, 方差的高斯分布。给定模型参数 ,可以很容易地定义最大似然(ML)过程来推断最佳形状和纹理参数:
  	      (4)
通过考虑先验分布的最大后验(MAP)来估计带正则化项的最优形状p和最优纹理参数c:
  (5)
公式5与公式1定义的优化问题等价。
\subsection{基于CLM的分割}
相对于AAM,ASM只使用特征点边缘灰度或轮廓线模型来进行点匹配,而CLM通过其形状标记点邻域内候选块来定义对象的纹理,同时利用与AAM类似的全局形状作为全局约束。针对初始化形状的各个标记点,用检测器对局部区域进行判别,作用类似滤波器,可获得激活得分响应图,标记点被正确对齐与否的概率可以定义为:
  	(6)
式中 指示定位正确与否,Ci是区分标记点xi对齐与否的分类器,可使用不同分类器,例如逻辑回归[9]、多通道相关滤波(MCCF)的平方误差总和最小滤波器(MOSSE)[18]和支持向量回归机(SVR)[16]等。
拟合CLM涉及到解决以下优化问题[19]: 
  	(7)
式中  ,Λ是计算与形状特征向量相关联的特征对角矩阵和ρ2是估计的形状噪声。
公式7可跟AAM一样改写为概率形式[20]:
  	(8)
已经提出了不同的方法仿真模拟真实的响应映射 ,最常用的是[19]的非参数方法(RLMS),它将真实的响应图近似为: 
	  	(9)
式中当前标记点位置xi是根据先前的概率生成形状模型定义的。将9代入8,得以下优化问题:
  	(10)
这相当于由8定义的优化问题,式中响应映射在所有像素位置  可评估 ,视真正的标记点位置yi作为潜在变量,式10可以使用EM算法迭代地求解[28]。。
 
 
图2 定性比较三种不同特征激活图及相应的局部响应映射图, MCCF通过多通道相关滤波器近似响应图,且用RLMS算法移动到最优位置。SVR基于支持向量机简单地选择最大响应位置。HG-n表示所用不同HG模块数的局部响应图,n取1,2,4。
Fig 2 Qualitative comparison between the three local detector strategies. The MCCF approximates the response map by multti-channel correlation filter and uses RLMSalgorithm to move to the nearest mode of the density. The SVR simply chooses the maximum detector response based on SVM. HG-n represents the number of different HG modules used to obtain the local response map, n take 1,2,4.
 
\section{小结与讨论}


