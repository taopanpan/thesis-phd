
\chapter{攻读学位期间发表的学术论文与科研成果}

\section*{已发表论文}
\begin{enumerate}

\item {\textbf{Pan~Tao},Zhongliang~Fu,Lili~Wang,Kai~Zhu.
{\href{http://link.springer.com/10.1007/978-981-10-3005-5_11}
{Perceptual Loss with Fully Convolutional for Image Residual Denoising.}
{ \textit{Pattern Recognition}}.\textbf{CCPR(EI)}. 2016. 122--132,DOI:10.1007/978-981-10-3005-5-11}}

\item {\textbf{陶攀},付忠良,朱锴,王莉莉.{金字塔分解的深度可视化方法},{哈尔滨工业大学学报(\textbf{EI})},2017,49(11):60-65,DOI:10.11918/j.issn.0367-6234.201612087}

\item{\textbf{陶攀},付忠良,朱锴.{基于深度学习的医学计算机辅助检测算法},{生物医学工程学杂志(\textbf{EI}),已录用},2017}
\item {\textbf{陶攀},付忠良,朱锴,王莉莉.{基于深度学习的超声心动图切面识别方法研究},{计算机应用(中文核心),2017.DOI:10.11772/j.issn.1001-9081.2017.05.1434} }

\item{Xianghu Ji,Lili~Wang,\textbf{Pan~Tao},Zhongliang~Fu.
{\href{http://link.springer.com/10.1007/978-981-10-3002-4_27}{Landmark Selecting on 2D Shapes for Constructing Point Distribution Model.}
,{ \textit{Pattern Recognition}},\textbf{CCPR(EI)} 2016, 318--331.DOI:10.1007/978-981-10-3002-4-27}}

\item {Lili~Wang,Zhongliang~Fu,\textbf{Pan~Tao}.{\href{http://ieeexplore.ieee.org/document/7529574}
{Four-chamber plane detection in cardiac ultrasound images based on improved imbalanced AdaBoost algorithm },
{ \textit{IEEE}},\textbf{ICCCBDA(EI)} 2016,299-303.DOI:10.1109/ICCCBDA.2016.7529574}}

\end{enumerate}
\section*{国家发明专利}
\begin{enumerate}
\item { 纪祥虎,高思聪,\textbf{陶攀},王莉莉.{用于统计形状模型的特征点辅助标注方法. {(申请号:201510672503.8)专利公开号:CN105205827 A}.{2015}}}
\end{enumerate}
\section*{项目经历}
\begin{enumerate}
\item {2015–2016}{四川省科技创新苗子工程—— 基于自动分割技术的左心室可视化及功能评价临床教学平台}{\textnormal{( 编号:2015060)}}{\newline 项目描述:本项目主要目标在于使用机器学习方法对左心室进行分割,得到左心室轮廓及结构和心功能参数;使用可视化技术对心脏左室进行三维立体结构教学。帮助麻醉医生学员快速学习掌握超声心动图中左心室结构}
{ \newline 项目职责:在项目中主要负责超声图像中心脏器官的自动定位和分割,分别利用机器学习的方法对超声图像中的左心室定位,和AAM方法对肾脏进行分割。 \newline 项目成果:形成论文两篇,专利一项,期间主要研究了基于深度学习的图像预处理方法,基于形状对齐模型进行心室分割,及基于深度级联回归模型进行心室边界分割算法等}{}  

\item{2015-2015}{ 阿里巴巴大规模图像搜索赛(38名共843支参赛队伍)}{\textnormal{\newline 本项目标是从海量图像中检索最相同或似的Top20图像 }}{ \newline 主要负责使用深度学习模型对图像进行特征抽取,同时配合队友进行图像检索等其他工作,其中用时一个月根据matconvnet写了一个C++版本的CNN框架的API,从中获得了处理百万级数据的经验,获得了使用OpenBLAS处理大型矩阵运算的经验}{\newline 项目收获:形成论文一篇,熟悉了深度学习提取语义特征进行实例检索的各项关键技术}{}

\item{2015-2017}{四川科技支撑计划--医学图像挖掘与心脏智能诊疗系统关键技术研究}{\textnormal{ \newline 项目描述:本项目主要目标在于使用机器学习方法对超声心动图标准切面进行自动识别。超声图像标准切面分类模块,包括图像预处理、特征提取和分类器模型构建实现标准切面自动识别分类;基于云端的海量切面视频的语义检索模块等}}
{\newline 项目职责:项目参与人 \newline 任务分工:图像预处理、特征提取、分类器建模、视频语义检索}{\newline 项目成果:发表论文三篇,分别研究了基于深度学习理论可视化分析其有效性,基于深度特征的超声图像标准切面自动识别算法等}{}

\item{2013-2014}{四川省科技支撑项目,华西医院合作项目–医学可视化模拟教学和诊断系统}{\textnormal{ \newline 项目描述:项目旨在为无经验的心脏外科医生和学员提供可视化的教学方案,同时通过机器学习和图形图像处理方法对三维心脏进行开放式建模,以提出一种基于心脏开放模型的智能诊疗综合系统}}
{\newline 项目职责:在项目中负责超声图像处理和基于机器学习的病理挖掘工作。 \newline 任务分工:图像预处理}{\newline 项目成果:参与撰写专利两项,对超声仪器,心脏疾病临床基本知识有较全面的了解;设计了针
对心脏超声图像的分割识别方法,以及病理挖掘方法;学习了基于偏微分方程的图像去噪和基于水平集的分割方法}{}

\end{enumerate}
\section*{在审和Working论文}
\begin{enumerate}
\item{\textbf{陶攀},付忠良.{基于Fast-rcnn的医学实例检索方法研究},{Working},2015}
\item{\textbf{陶攀},付忠良.{基于超声心动图的左心室分割综述},{Working},2015}
\item{\textbf{陶攀},付忠良.{基于形状对齐的超声心动图左心室分割方法},{工程科学学报(在审)},2017}
\item{\textbf{陶攀},付忠良.{基于形状对齐的超声心动图左心室分割方法},{工程科学学报(在审)},2017}
\item{\textbf{陶攀},付忠良.{基于CNN-LSTM的超声心动图左心室分割方法},{Working},2017}

\end{enumerate}


\section*{参与项目编写和申请}
\begin{enumerate}
\item{2016}{四川科技支撑计划--医学图像挖掘与心脏智能诊疗系统关键技术研究 }{}
\item{2016}{基于医学图像建模的心功能评价系统研发与应用}{}
\item{2015}{国科控股技术创新项目--交互式视觉仿真关键技术研究与产品应用示范}{}
\item{ 2014}{西部之光项目--基于医学图像建模的评价系统 }{}
\item{2014}{数字化医疗辅助设备关键技术研发—基于机器智能的三维可视化手术诊疗仿真平台}{}
\end{enumerate}
\section*{获奖及荣誉}
\begin{enumerate}
\item{2015}{中国科学院研究生院“三好学生”荣誉称号}
\item{2016}{中国科院大学优秀学生干部}
\item{2017}{中国科学院博士国家奖学金}
\end{enumerate}
%%
%%% >>> Acknowledgements
%%

\chapter{致\quad 谢}

转眼博士求学生涯即将结束,我要衷心感谢所有关心爱护我、帮助支持我的老师同学、好友以及家人。

首先,我要感谢我的导师付忠良研究员,在博士四年及硕士两年期间,付老师以其广博的知识、耐心指导学生的人格魅力、严谨的治学态度以及创新的科学精神深深地影响了我,在科学研究以及日常生活各方面都给予我最大的支持,不仅悉心传授我专业知识,更重要的是注重培养我做科研以及创新的能力,在教授理论知识的同时,注重理论联系实际。同时,他以身作则的态度给我树立了良好的榜样,在培养我专业技能的同时注重人格的培养,使我真正成为一个对社会有用的人。这些将使我终生受益。

感谢四川大学华西医院的宋海波医生以及其助手,使我对医学图像处理产生浓厚的兴趣,非常感谢姚宇老师及成都计算机所各位老师给我提供良好的学习环境,感谢他们对我学术研究的帮助和指导。

感谢我父亲陶朝重和母亲陶爱湘的养育之恩,父母一辈子务农,辛苦抚养我们兄妹两个长大,谨以本文给我最敬爱的父亲!

最后向参加论文评审和答辩的专家老师们表示感谢!