%%
%%% >>> Title Page
%%
%%
%%% Chinese Title Page
%%
  \confidential{}% show confidential tag
  \schoollogo{scale=4.2}{ucas_logo}% university logo
  \title[中国科学院大学博士论文]{基于深度学习的医学图像内容理解关键技术研究}% \title[short title for headers]{Long title of thesis}
  \author{陶 攀}% name of author
  \advisor{付忠良~研究员}% names and titles of supervisors
  \degree{博士}% degree
  \degreetype{工学}% degree type
  \major{计算机软件与理论}% major
  \institute{中国科学院成都计算机应用研究所}% institute of author
  %\chinesedate{2014~年~06~月}% only need for user customized date
%%
%%% English Title Page
%%
  \englishtitle{Research on Key Technologies in Medical Image Processing \\  Based on Deep Learning }
  \englishauthor{Pan Tao}
  \englishadvisor{Supervisor: Professor Zhongliang FU}
  \englishdegree{Doctor of Engineering}
  \englishthesistype{thesis}
  \englishmajor{Computer Software and Theory}
  \englishinstitute{Chengdu Institute of Computer Applications \\ Chinese Academy of Sciences}
  %\englishdate{June, 2014}% only need for user customized date
%%
%%% Generate Chinese Title
%%
\maketitle
%%
%%% Generate English Title
%%
\makeenglishtitle
%%
%%% >>> Author's declaration
%%
\makedeclaration
%%
%%% >>> Abstract
%%
\chapter*{摘\quad 要}% does not show the title on the top
\chaptermark{摘\quad 要}
\setcounter{page}{1}% set page number
\pagenumbering{Roman}% set large roman

对医学图像的内容理解是应用计算机视觉与人工智能进行医学影像分析的最基本问题之一,从二维或三维图像数据中理解图像内容一直是医学图像应用研究的重点领域,涉及到感兴趣目标的去噪、分类、检测、分割和检索等研究内容。由于图像内容理解问题本身的困难性,并且医学图像存在特有的领域先验,如超声特有的斑点噪声,衰减,阴影等因素影响,导致目标受尺度、旋转、形变等而形成不同的成像,使得用计算机对医学图像中的内容进行鲁棒的表达与识别依然是一个严峻的挑战。主要原因之一是不同具体任务分别处于图像抽象的的不同层次,如何有效结合低层的图像数据信息和高层的语义信息是解决医学图像的内容理解问题的关键所在。

深度学习为代表的人工智能技术在医学影像分析领域呈现出了非常引人注目的研究进展。但仍有一系列问题难以克服:(1)不同网络结构对特征学习的影响(如何设计高效结构理论可解释性;(2)学习到的特征为什么能起作用?可视化;(3)深度学习的优异泛化能力从何而来?检测分割。

针对以上问题,论文工作主要涉及深度卷积神经网络自动提取分层递阶图像特征,并在不同层次特征应用到不同抽象层次图像内容理解的研究,论文在深入分析传统计算机视觉算法的基础上,重点研究了深度学习模型、基于形状先验的统计形状模型,并致力于利用形状先验信息结合深度卷积神经网络解决医学特定目标检测与分割问题。

论文的主要工作和创新之处在于:

通过构建标准切面数据库,提出了一种基于深度卷积神经网络的超声心动图标准切面自动识别方法,该算法针对网络全连接层占有模型大部分参数的缺点,引入空间金字塔均值池化替代全连接层,获得更多的空间结构信息,利用全局空间金字塔均值池化方法进行微调迁移学习,并大大减少模型参数、降低过拟合风险,通过类别显著性区域将类似注意力机制引入模型可视化过程,详尽分析了数据规模对模型分类精度的影响,并对模型的可解释性和有效性进行了分析。

针对基于深度卷积神经网络的图像分类模型的可解释性问题,通过评估模型特征空间的潜在可表示性,提出一种用于改善理解模型特征空间的可视化方法。给定任何已训练的深度卷积网络模型,引入了通过激活最大化获得的图像可解释性的正则化方法,结合现有正则化方法提出空间金字塔分解方法,利用构建多层拉普拉斯金字塔主动提升目标图像特征空间的低频分量,结合多层高斯金字塔调整其特征空间的高频分量得到较优可视化效果。并通过限制可视化区域,提出利用类别显著性激活图技术加以压制上下文无关信息,可进一步改善可视化效果。该模型有效克服了原有可视化方法中由于不能主动调整高低频分量等原因造成的可视化图像语义重复和低效率等问题。

针对自动检测医学图像中指定目标时存在的问题,提出了一种基于深度学习自动检测目标位置和估计对象姿态的算法。该算法基于区域深度卷积神经网络和目标结构的先验知识,采用区域生成候选框网络、感兴趣区域池化策略,引入包括分类损失、边框位置回归定位损失和像平面内朝向损失的多任务损失函数,近似优化一个端到端的有监督定位网络,能快速地对医学图像中目标自动定位,有效地为下一步的分割和参数自动提取提供定位结果。并在超声心动图左心室检测中提出利用检测额外标记点:二尖瓣环、心内膜垫和心尖,能高效地对左心室朝向姿态进行估计。

针对图像去噪中存在的问题,我们提出了一个有监督多层残差卷积网络框架,结合不同损失函数学习端到端映射变换。输入是带噪声的图像和原图像,输出的是去噪后的图像。

针对医学超声图像的对比度低、存在斑点噪声导致难以分割的问题,提出一种利用沙漏卷积神经网络特征的多尺度形状模型分割方法,自动定位经食道超声心动图中心室并全自动分割心室内外膜。首先,结合梯度方向直方图特征和支持向量机的心室自动检测方法,自动确定分割模型中的初始轮廓;其次, 将心室分割任务纳入统计形变模型形状特征点对齐任务框架,通过比较不同外观纹理特征和激活图,包括传统手工设计的特征和利用深度学习自动学习的卷积特征,提出利用堆叠多级沙漏卷积网络建模心室外观的全局和局部信息,统一活动外观模型和局部受限模型的概率形式,采用反向组合梯度下降算法迭代优化分割结果,完成左心室轮廓的自动提取。然后,以医生手动勾勒的轮廓作为“金标准”,通过构造心室分割数据集以评价算法,且提出了扩充数据样本的方法来克服深度卷积网络过拟合问题,进行详尽实验讨论分析了基于不同层级的多级沙漏卷积网络对全局和局部纹理特征建模能力对分割效果的影响。实验结果表明,卷积模块允许网络提取专门用于指定任务的特征,并通过实验显示其优于手工设计的特征。该方法分割效果优于传统形状对齐方法,能够解决自动定位超声心动图中左心室的初始轮廓和弱边界自动分割的问题。


\keywords{深度学习,卷积神经网络,图像内容理解}
%\end{abstract}


\chapter*{Abstract}% does not show the title on the top
\chaptermark{Abstract}
%\begin{englishabstract}% will show the title on the top
Machine learning is used in the medical imaging field, including computer-aided diagnosis, image
segmentation, image registration, image fusion, image-guided therapy, image annotation,
and image database retrieval. Deep learning methods are a set of algorithms in machine learning,
which try to automatically learn multiple levels of representation and abstraction that help
make sense of data. This in turn leads to the necessity of understanding and examining the characteristics
of deep learning approaches, in order to be able to apply and refine the methods in a
proper way.

The aim of this work is to evaluate deep learning methods in the medical domain and to
understand if deep learning methods (random recursive support vector machines, stacked sparse
auto-encoders, stacked denoising auto-encoders, K-means deep learning algorithm) outperform
other state of the art approaches (K-nearest neighbor, support vector machines, extremely randomized
trees) on two classification tasks, where the methods are evaluated on a handwritten
digit (MNIST) and on a medical (PULMO) dataset. Beside an evaluation in terms of accuracy
and runtime, a qualitative analysis of the learned features and practical recommendations for the
evaluated methods are provided within this work. This should help improve the application and
refinement of the evaluated methods in future.

Results indicate that the stacked sparse auto-encoder, the stacked denoising auto-encoder
and the support vector machine achieve the highest accuracy among all evaluated approaches on
both datasets. These methods are preferable, if the available computational resources allow to
use them. In contrast, the random recursive support vector machines exhibit the shortest training
time on both datasets, but achieve a poorer accuracy than the afore mentioned approaches. This
implies that if the computational resources are limited and the runtime is an important issue, the
random recursive support vector machines should be used.

\englishkeywords{University of Chinese Academy of Sciences (UCAS)}
%\end{englishabstract}
