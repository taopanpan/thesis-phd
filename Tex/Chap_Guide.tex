
\chapter{使用简介}
\label{chap:guide}

为方便使用及更好地展示\LaTeX{}排版的优秀特性,本人对模板的框架和文件体系进行了细致地处理,尽可能地对各个功能和板块进行了模块化和封装,对于初学者来说,众多的文件目录也许会让人觉得有些无所适从,但阅读完下面的使用说明后,您会发现原来使用思路是简单而清晰的,而且,当对\LaTeX{}有一定的认识和了解后,会发现其相对Word类排版系统的极具吸引力的优秀特性。所以,如果您是初学者,请不要退缩,请稍加尝试和坚持,让自己领略到\LaTeX{}的非凡魅力,并可以通过阅读相关资料如Wikibook\citep{wikibook2014latex}来完善自己的使用知识。

\section{先试试效果}

\begin{enumerate}
    \item 安装软件:根据所使用的操作系统和章节~\ref{sec:system}中的信息安装\LaTeX{}编译环境。
    \item 获取模板:下载 \href{https://github.com/mohuangrui/ucasthesis}{ucasthesis} 模板并解压。ucasthesis模板不仅只是提供了相应的类文件,同时也提供了包括参考文献等在内的完成学位论文的一切要素,所以,下载时,推荐下载整个ucasthesis文件夹,而不是单独的文档类。
    \item 编译模板:
        \begin{enumerate}
            \item Windows用户:双击运行artratex.bat脚本。
            \item Linux或Mac OS用户: 打开 \verb|terminal| -> 运行 \verb|chmod +x ./artratex.sh| -> 运行 \verb|./artratex.sh xa|
        \end{enumerate}
    \item 处理错误:若编译中遇到了问题,请先查看“常见问题”(章节~\ref{sec:qa})。
\end{enumerate}

编译完成后,即可获得本PDF说明文档。而这也完成了学习使用此模板撰写论文的一半进程。什么?这就学成一半了,这么简单???,是的,就这么简单!

\section{文档目录简介}

\subsection{Thesis.tex}

Thesis.tex为主文档,其设计和规划了论文的整体框架,通过对其的阅读可以让用户了解整个论文框架的搭建。

\subsection{编译脚本}

\begin{itemize}
    \item Windows用户:双击Dos脚本artratex.bat可得全编译后的PDF文档。
    \item Linux或Mac OS用户:在terminal中运行
        \begin{enumerate}
            \item \verb|./artratex.sh xa|:获得全编译后的PDF文档
            \item \verb|./artratex.sh x|:快速编译模式
        \end{enumerate}
    \item 全编译是指运行 \verb|latex引擎+bibtex+latex引擎+latex引擎| 以正确生成所有的引用链接,如目录,参考文献及引用等。当文章在写作过程中,并无添加新的引用,则可用快速编译,即只运行一遍\LaTeX{}编译引擎以减少编译时间。
\end{itemize}


\subsection{Tmp文件夹}

运行编译脚本后,编译所生成的文档皆存于Tmp文件夹内,包括编译得到的PDF文档,其存在是为了保持工作空间的整洁,因为好的心情是很重要的。

\subsection{Style文件夹}

Style文件夹内包含ucasthesis文档类的定义文件和配置文件,对于有特殊需求的用户,通过对它们的修改可以实现特定的类设定。用户若需更新模板,一般只需用新的样式文件替换旧的即可。

\begin{enumerate}
    \item ucasthesis.cls:文档类定义文件,论文的最核心的格式即通过它来定义的。
    \item ucasthesis.cfg:文档类配置文件,设定如目录显示为“目~录”而非“目录”。
    \item artratex.sty: 常用宏包的加载及文档的设定,如参考文献样式,文献引用样式,页眉页脚设定等。模板为这些功能提供了开关选项,从而只需在Thesis.tex中的\verb+\usepackage[options]{artratex}+中进行启用即可,一般无需修改artratex.sty本身。
    \item artracom.sty: 用户自定义命令以及添加宏包的推荐放置位置。
\end{enumerate}

\subsection{Tex文件夹}

Tex文件夹内为论文的所有实体内容,正常情况下,这也是你\textbf{使用此模板撰写学文论文时,主要关注和修改的一个位置,注:所有文件都必须采用UTF-8编码,否则编译后将出现乱码文本},详细分类介绍如下:

\begin{itemize}
    \item Frontpage.tex:为论文封面内容及中英文摘要。
    \item Mainmatter.tex:索引需要出现的Chapter。开始写论文时,可以只索引当前章节,以快速编译查看,当论文完成后,再对所有章节进行索引即可。
    \item Chap{\_}xxx.tex:为论文主体的各个章节,可根据需要添加和撰写。
    \item Appendix.tex:为附录内容
    \item Backmatter.tex:为发表文章信息,致谢部分等。
\end{itemize}

\subsection{Img文件夹}

Img文件夹用于放置论文中所需要的图类文件,支持格式有:.jpg, .png, .pdf。其中,\verb|ucas_logo.pdf|为国科大校徽。不建议为各章节图片建子目录,即使图片众多,若命名规则合理,图片查询亦是十分方便。

\subsection{Biblio文件夹}

\begin{enumerate}
    \item ref.bib:考文献信息库。
    \item gbt7714-xxx.bst:符合国标的文献样式定义文件。

        由 \href{https://github.com/zepinglee/gbt7714-bibtex-style}{zepinglee}  开发,并满足最新国标要求。不建议尝试修改文献样式,若坚持,请查阅开发者所提供的文档。
\end{enumerate}

\section{数学公式、图表、参考文献等功能}

\subsection{数学公式}

Navier-Stokes方程:
\begin{equation} \label{eq:ns}
    \begin{cases}
        \frac{\partial \rho}{\partial t} + \nabla\cdot(\rho\Vector{V}) = 0 \\
        \frac{\partial (\rho\Vector{V})}{\partial t} + \nabla\cdot(\rho\Vector{V}\Vector{V}) = \nabla\cdot\Tensor{\sigma}\\
        \frac{\partial (\rho E)}{\partial t} + \nabla\cdot(\rho E\Vector{V}) = \nabla\cdot(k\nabla T) + \nabla\cdot(\Tensor{\sigma}\cdot\Vector{V})
    \end{cases}
\end{equation}

常用数学公式的命令代码模板,请见 \href{https://en.wikibooks.org/wiki/LaTeX/Mathematics}{WiKibook Mathematics}.

\subsection{表格}

请见这是一个样表(表~\ref{tab:sample})
\begin{table}[!htbp]
    \centering
    \footnotesize% fontsize
    \setlength{\tabcolsep}{4pt}% column separation
    \renewcommand{\arraystretch}{1.2}%row space 
    \begin{tabular}{lcccccccc}
        \hline\hline
        Row number & \multicolumn{8}{c}{This is a multicolumn} \\
        %\cline{2-9}% partial hline from column i to column j
        \hline
        Row 1 & $1$ & $2$ & $4$ & $5$ & $6$ & $7$ & $8$\\
        \hline
        Row 2 & $1$ & $2$ & $4$ & $5$ & $6$ & $7$ & $8$\\
        \hline
        Row 3 & $1$ & $2$ & $4$ & $5$ & $6$ & $7$ & $8$\\
        \hline
        Row 4 & $1$ & $2$ & $4$ & $5$ & $6$ & $7$ & $8$\\
        \hline\hline
    \end{tabular}
    \caption{这是一个样表。}
    \label{tab:sample}
\end{table}

\subsection{图片插入}

论文中图片的插入通常分为单图和多图,下面分别加以介绍:

单图插入:假设插入名为\verb|tc_q_criteria|(后缀可以为.jpg、.png、.pdf,下同)的图片,其效果如图\ref{fig:tc_q_criteria},其命令可为:
\begin{verbatim}
\begin{figure}[!htbp]
    \centering
    \includegraphics[width=0.45\textwidth]{tc_q_criteria}
    \caption{Q判据等值面图}
    \label{fig:tc_q_criteria}
\end{figure}
\end{verbatim}
\begin{figure}[!htbp]
    \centering
    \includegraphics[width=0.45\textwidth]{tc_q_criteria}
    \caption{Q判据等值面图}
    \label{fig:tc_q_criteria}
\end{figure}

如果插图的空白区域过大,希望减少插入图片后的留白,以图片\verb|shock_cyn|为例(图\ref{fig:shock_cyn}),可以使用如下代码模板:
\begin{verbatim}
\begin{figure}[!htbp]
    \centering
    %trim option's parameter order: left bottom right top
    \includegraphics[trim = 30mm 0mm 30mm 0mm, clip,
    width=0.40\textwidth]{shock_cyn}
    \caption{Shock diffraction}
    \label{fig:shock_cyn}
\end{figure}
\end{verbatim}
\begin{figure}[!htbp]
    \centering
    %trim option's parameter order: left bottom right top
    \includegraphics[trim = 30mm 0mm 30mm 0mm, clip, width=0.40\textwidth]{shock_cyn}
    \caption{激波圆柱作用。}
    \label{fig:shock_cyn}
\end{figure}

多图的插入如图\ref{fig:oaspl},其代码如下。
\begin{verbatim}
\begin{figure}[!htbp]
    \centering
    \begin{subfigure}[b]{0.45\textwidth}
      \includegraphics[width=\textwidth]{oaspl_a}
      \caption{}
      \label{fig:oaspl_a}
    \end{subfigure}%
    ~%add desired spacing
    \begin{subfigure}[b]{0.45\textwidth}
      \includegraphics[width=\textwidth]{oaspl_b}
      \caption{}
      \label{fig:oaspl_b}
    \end{subfigure}
    \begin{subfigure}[b]{0.45\textwidth}
      \includegraphics[width=\textwidth]{oaspl_c}
      \caption{}
      \label{fig:oaspl_c}
    \end{subfigure}%
    ~%add desired spacing
    \begin{subfigure}[b]{0.45\textwidth}
      \includegraphics[width=\textwidth]{oaspl_d}
      \caption{}
      \label{fig:oaspl_d}
    \end{subfigure}
    \caption{总声压级。(a)$A$,(b)$B$,(c)$C$,(d)$D$}
    \label{fig:oaspl}
\end{figure}
\end{verbatim}
\begin{figure}[!htbp]
    \centering
    \begin{subfigure}[b]{0.45\textwidth}
      \includegraphics[width=\textwidth]{oaspl_a}
      \caption{}
      \label{fig:oaspl_a}
    \end{subfigure}%
    ~%add desired spacing
    \begin{subfigure}[b]{0.45\textwidth}
      \includegraphics[width=\textwidth]{oaspl_b}
      \caption{}
      \label{fig:oaspl_b}
    \end{subfigure}
    \begin{subfigure}[b]{0.45\textwidth}
      \includegraphics[width=\textwidth]{oaspl_c}
      \caption{}
      \label{fig:oaspl_c}
    \end{subfigure}%
    ~%add desired spacing
    \begin{subfigure}[b]{0.45\textwidth}
      \includegraphics[width=\textwidth]{oaspl_d}
      \caption{}
      \label{fig:oaspl_d}
    \end{subfigure}
    \caption{总声压级。(a)$A$,(b)$B$,(c)$C$,(d)$D$}
    \label{fig:oaspl}
\end{figure}

撰写论文中,插图和制表常用到的命令,已在\verb|Tex/Commands.tex|这个文本中给出了参考代码,大家只需拷贝使用即可。

\subsection{参考文献引用}

参考文献引用过程以实例进行介绍,假设需要引用名为"Document Preparation System"的文献,步骤如下:

1)使用Google Scholar搜索Document Preparation System,在目标条目下点击Cite,展开后选择Import into BibTeX打开此文章的BibTeX索引信息,将它们copy添加到ref.bib文件中(此文件位于Biblio文件夹下)。

2)你会发现索引信息中第一行为 \verb|@article{lamport1986document,|。其中 \verb|lamport1986document| 即为此文献的label (\textbf{中文文献也必须使用英文label},一般遵照:姓氏拼音+年份+标题第一字拼音的格式),想要在论文中索引此文献,有两种索引类型:

文本类型:\verb|\citet{lamport1986document}|。正如此处所示 \citet{lamport1986document}; 

括号类型:\verb|\citep{lamport1986document}|。正如此处所示 \citep{lamport1986document}。

\textbf{多文献索引用英文逗号隔开}:

\verb|\citep{lamport1986document,chen2005zhulu}|。正如此处所示 \citep{lamport1986document,chen2005zhulu}

如此,即完成了文献的索引,请查看下本文档的参考文献一章,看看是不是就是这么简单呢?是的,就是这么简单!

不同文献样式和引用样式可在Thesis.tex中对artratex.sty调用实现,如:
\begin{itemize}
    \footnotesize
    \item \verb+\usepackage[numbers]{artratex}+ $\%$ 文本: Jones [1]; 括号: [1]
    \item \verb+\usepackage[super]{artratex}+ $\%$ 文本: Jones 上标[1]; 括号: 上标[1]
    \item \verb+\usepackage[authoryear]{artratex}+ $\%$ 文本: Jones (1995); 括号: (Jones, 1995)
    \item \verb+\usepackage[alpha]{artratex}+ $\%$ 文本: 不可用; 括号: [Jon95]
\end{itemize}

若在上标(super)模式下,希望在特定位置将上标改为嵌入式标,可使用

文本类型:\verb|\citetns{lamport1986document,chen2005zhulu}|。

正如此处所示\citetns{lamport1986document,chen2005zhulu}

括号类型:\verb|\citepns{lamport1986document,chen2005zhulu}|。

正如此处所示\citepns{lamport1986document,chen2005zhulu}

参考文献索引更为详细的信息,请见Wikibook\citep{wikibook2014latex} \nocite{*}。

\section{常见使用问题}\label{sec:qa}

\begin{enumerate}
    \item 模板每次发布前,都已在Windows,Linux,MacOS系统上测试通过。下载模板后,若编译出现错误,则请遵从 \href{https://github.com/mohuangrui/ucasthesis}{位于主页底部的用户指南}。

    \item 模板文档的编码为UTF-8编码。所有文件都必须采用UTF-8编码,否则编译后生成的文档将出现乱码文本。若出现文本编辑器无法打开文档或打开文档乱码的问题,请检查您使用的编辑器对UTF-8编码的支持。如果使用WinEdt作为文本编辑器(不推荐使用),应在其Options -> Preferences -> wrapping选项卡下将两种Wrapping Modes中的内容:TeX;HTML;ANSI;ASCII|DTX...修改为:TeX;\textbf{UTF-8|ACP;}HTML;ANSI;ASCII|DTX...同时,取消Options -> Preferences -> Unicode中的Enable ANSI Format...选项。

    \item 推荐选择xelatex或lualatex编译引擎编译中文文档。编译脚本的默认设定为xelatex编译引擎。你也可以选择不使用脚本编译,如直接使用 \TeX{}文本编辑器编译。注:\TeX{}文本编辑器编译的默认设定为pdflatex编译引擎,若选择xelatex或lualatex编译引擎,请进入下拉菜单选择。为正确生成引用链接,需要进行全编译。
    \item Texmaker使用简介
        \begin{enumerate}
            \footnotesize
            \item 使用 Texmaker “打开” Thesis.tex。
            \item 菜单 “选项 (Options)” -> “设置当前文档为主文档 (Define as Master Document)”
            \item 菜单 “自定义 (User)” -> “自定义命令 (User Commands)” -> “编辑自定义命令 (Edit User Commands)” -> 左侧选择 “command 1”,右侧 “菜单项 (Menu Item)” 填入 Auto Build -> 点击下方“向导 (Wizard)” -> “添加 (Add)”: xelatex + bibtex + xelatex + xelatex + pdf viewer -> 点击“完成 (OK)”
            \item 使用 Auto Build 编译带有未生成引用链接的源文件,可以仅使用 xelatex 编译带有已经正确生成引用链接的源文件。
            \item 编译完成,“查看(View)” PDF,在pdf中 “ctrl+click” 可链接到相对应的源文件。
        \end{enumerate}
    
    \item 模版的设计可能地考虑了适应性。致谢等所有条目都是通过最为通用的

        \verb+\chapter{item name}+  and \verb+\section*{item name}+

        来显式实现的 (请观察Backmatter.tex),从而可以随意添加,放置,和修改,如同一般章节。对于图表目录名称则可在ucasthesis.cfg中进行修改。

    \item 设置文档样式: 在artratex.sty中搜索关键字定位相应命令,然后修改
        \begin{enumerate}
            \item 正文行距:修改 \verb|\linespread{1.3}|
            \item 参考文献行距:修改 \verb|\setlength{\bibsep}{0.0ex}|
            \item 目录显示subsection:修改 \verb|\setcounter{tocdepth}{2}|
            \item 文档内超链接的颜色及其显示:修改 \verb|\hypersetup|
            \item 页眉页脚的设定:frontmatterstyle,mainmatterstyle,和backmatterstyle分别用于定义前言,主要内容,和附录的页眉页脚样式。通过阅读这一部分的代码,可以轻松地理解和修改以获得自定义的样式。命令的详细解释请参见 \href{https://www.ctan.org/pkg/fancyhdr?lang=en}{fancyhdr} 的用户文档。同时可参见 \href{https://ctan.org/pkg/ctex?lang=en}{ctex} 宏包用户文档。

            \item 设置图2.3为图2-3: 设置
                {
                    \footnotesize
\begin{verbatim}
\renewcommand{\theequation}{\arabic{chapter}-\arabic{equation}}
\renewcommand{\thefigure}{\arabic{chapter}-\arabic{figure}}
\renewcommand{\thetable}{\arabic{chapter}-\arabic{table}}
\end{verbatim}
                }
        \end{enumerate}

    \item 字体控制。文档内字体切换方法:
        \begin{itemize}
            \item 宋体:飞扬跋扈~或 \textrm{飞扬跋扈}
            \item 粗宋体:{\bfseries 飞扬跋扈} 或 \textbf{飞扬跋扈}
            \item 黑体:{\sffamily 飞扬跋扈} 或 \textsf{飞扬跋扈}
            \item 粗黑体:{\bfseries\sffamily 飞扬跋扈} 或 \textsf{\bfseries 飞扬跋扈}
            \item 仿宋:{\ttfamily 飞扬跋扈} 或 \texttt{飞扬跋扈}
            \item 楷体:{\itshape 飞扬跋扈} 或 \textit{飞扬跋扈}
        \end{itemize}
        
        由于缺乏一个统一的被各个操作系统所默认携带的完备的中文字体库,\href{https://ctan.org/pkg/ctex?lang=en}{ctex} 针对不同的操作系统而调用各系统上所对应的一类中文字体库。由于很多操作系统的字库往往缺乏原生态的加粗宋体字重,有时会发生加粗宋体被黑体所替换的情形,这对封面的字体造成影响。若需要解决这个问题,可采用调用自定义的一个完备字体库的方案。
        
        若需设置字体库,请选择xelatex或lualatex编译引擎,并设置需要的字体库。如用Times New Roman作为英文字体,在artratex.sty中设置:

                 \verb|\setmainfont{Times New Roman}|

                 如果需要调用一个自定义的中文字体库,方法为:

                \begin{itemize}
                    \item 调用 \href{https://ctan.org/pkg/ctex?lang=en}{ctex} 预定义好的备用字库: 在Thesis.tex中设置

                        {
                            \small
                            \verb|\documentclass[doublesided,fontset=fandol]{Style/ucasthesis}%|
                         }

                         便可方便地调用 \href{https://ctan.org/tex-archive/fonts/fandol?lang=en}{fandol} 这一字体库。\LaTeX{}编译系统一般已携带或是能自动下载安装 \href{https://ctan.org/tex-archive/fonts/fandol?lang=en}{fandol} 字库。若不能,则请手动下载并安装链接所提供的所有字体即可。本模板使用说明文档就是采用调用 \href{https://ctan.org/tex-archive/fonts/fandol?lang=en}{fandol} 中文字库。

                     \item 手动调用系统带有的中文字库: 首先需要查看系统所带有的中文字库及其名称,也可选择安装可获得的中文字库。假设系统已安装或带有名为SC的中文字库(此字库为MacOS所配备,具备原生态的加粗宋体),则可在artratex.sty的

                         \verb|\RequirePackage{fontspec}|

                        行下添加如下中文字体调用命令:
                        {
                            \scriptsize
\begin{verbatim}
\setCJKmainfont[BoldFont=Songti SC Bold,ItalicFont=Kaiti SC]{Songti SC Light}%
\setCJKsansfont{Heiti SC}%
\end{verbatim}
                        }
                \end{itemize}

                字库调用的全面解释可参见 \href{https://ctan.org/pkg/fontspec}{fontspec} (英文字体调用)和 \href{https://ctan.org/pkg/xecjk?lang=en}{xeCJK} (中文字体调用)。因为模版的设定考虑兼顾不同操作系统(Windows, Linux, Mac OS),为了模版的健壮性,上述字体设置和调用方案并未作为原始设定。
            \item 如果对封面下划线上的文本不居中下划线产生疑问,这是因为那些下划线前面还有字头,导致文本只能在页面居中和在下划线上居中二选一。当前封面采取页面居中。如需要调整文本在下划线上的位置,可用 \verb|\hspace{+/- n.0em}| 命令来插入或删除 n 个空格,进行手动调整。这个解决方案是很不优雅,但问题的本质还是模板样式的不合理。
    \item  一般规范下,章应开始于奇数页。从而若前一章结束于奇数页,则一空白页将被插入以保证上述规则。如想修改以取消空白页,有如下方案:
     \begin{itemize}
         \item 在thesis.tex的documentclass中用singlesided替代doublesided。这使文档不区分奇偶页,因此章可以开始于任意页。此方案将移除所有空白页,包括封面处的。同时,页眉页脚的设定不再区分奇偶页。
         \item 可以在ucasthesis.cls文件中,将cleardoublepage命令的定义修改为:

             \verb|\def\cleardoublepage{\clearpage}|

             这一命令使产生空白页的机制失效。这一方案将移除所有的空白页,包括封面处的。但与方案一不同的是,页面页脚的设定可以区分奇偶页。
         \item 在thesis.tex的documentclass中添加openany选项(openany与doublesided和printcopy都可搭配)。这一命令使章可以开始于任意页。同时,将artratex.sty中和thesis.tex中的cleardoublepage改为clearpage。此方案将移除所有的用于调整章的起始位置的空白页,而不包括封面处的。同时,页面页脚的设定可以区分奇偶页。
     \end{itemize}
      无论哪种方案都要注意对页眉页脚的影响并做出合适的调整。推荐是采用默认设置,尽量避免将精力花在这些无关紧要的细节上。\LaTeX{}的特点是标准化,而其导致的问题则是任何脱离标准的修改都将花费相当精力。对于电子档的论文,在thesis.tex的documentclass中,若不想使用doublesided,则可使用singlesided来减少空白页。而对于打印版,启用printcopy选项以替换doublesided/singlesided选项,这样可使奇偶页的排版在打印装订后更美观。

  \item 部分所也许对论文格式有不同的设定,\LaTeX{}用户亦无需担心,仍可放心地使用当前模板进行论文撰写,因为\LaTeX{}的特征就在于内容与格式的分离。在使用此模板完成论文的撰写后,任何形式的格式调整都可独立于内容进行,并可只需通过修改模板样式文件中的少数命令轻松快速完成,并无风险。
\end{enumerate}


