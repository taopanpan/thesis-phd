\chapter{总结与展望}
\label{chap:Sum}

现有的医学图像分析方法和相关的研究内容具有内容博杂、应用对象复杂、专业性强的特点。由于医学图像的本身复杂性,不同模态图像成像原理不同,即使同一部位的不同模态的图像内容都不一致,目前没有一种对所有模态医学图像都适用的全自动分析方法以满足临床需求。而深度学习的出现打破了现状,深度CNN在图像处理、视频、语音识别和自然语言文本处理中均取得了突破。本文中我们主要从计算机视觉的角度对最近CNN在医学图像领域取得的进展进行了研究。首先概述了深度学习在医学图像领域的研究现状,其中分别介绍了不同应用领域的研究现状,还介绍了相关软硬件平台的发展情况,并给出了当前的趋势和面临的挑战;然后讨论了CNN在不同方面取得的进步的相关基础理论:比如层的设计结构、激活函数、损失函数、正则化、优化等方面的基础知识;除了从CNN的各个方面回顾其进展,我们结合CNN在计算机视觉任务上应用到医学图像分析领域的具体任务,其中包括医学超声图像分类、超声图像和磁共振图像中心室的检测以及超声心动图心室分割;并从分类识别引申出对深度模型的可解释性问题的研究,即深度可视化分析。将分别详细总结研究内容并展望相关研究方向。

\section{研究总结}

针对特征表示的高层语义识别问题,构建超声心动图的标准切面数据库,提出了一种基于深度卷积神经网络的自动识别方法,该算法针对网络全连接层占有模型大部分参数的缺点,引入空间金字塔均值池化替代全连接层,获得更多的空间结构信息,利用全局空间金字塔均值池化方法进行微调迁移学习,并大大减少模型参数、降低过拟合风险,同时通过类别显著性区域将类似注意力机制引入模型可视化过程,详尽分析了数据规模对模型分类精度的影响,并对模型的可解释性和有效性进行了分析。

针对基于深度卷积神经网络的图像分类模型的可解释性问题,通过评估模型特征空间的潜在可表示性,提出一种用于改善理解模型特征空间的可视化方法。给定任何已训练的深度卷积网络模型,引入了通过激活最大化获得的图像可解释性的正则化方法,结合现有正则化方法提出空间金字塔分解方法,利用构建多层拉普拉斯金字塔主动提升目标图像特征空间的低频分量,结合多层高斯金字塔调整其特征空间的高频分量得到较优可视化效果。并通过限制可视化区域,提出利用类别显著性激活图技术加以压制上下文无关信息,可进一步改善可视化效果。该模型有效克服了原有可视化方法中由于不能主动调整高低频分量等原因造成的可视化图像语义重复和低效率等问题。

针对自动检测医学图像中指定目标时存在的问题,提出了一种基于深度学习自动检测目标位置和估计对象姿态的算法。该算法基于区域深度卷积神经网络和目标结构的先验知识,采用区域生成候选框网络、感兴趣区域池化策略,引入包括分类损失、边框位置回归定位损失和像平面内朝向损失的多任务损失函数,近似优化一个端到端的有监督定位网络,能快速地对医学图像中目标自动定位,有效地为下一步的分割和参数自动提取提供定位结果。并在超声心动图左心室检测中提出利用检测额外标记点:二尖瓣环、心内膜垫和心尖,能高效地对左心室朝向姿态进行估计。

针对特征表示的底层视觉任务:图像去噪和分割中存在的问题,我们提出了一个有监督多层残差卷积网络框架,结合不同损失函数学习端到端映射变换;针对医学超声图像的对比度低、存在斑点噪声导致难以分割的问题,提出一种利用沙漏卷积神经网络特征的多尺度形状模型分割方法,自动定位经食道超声心动图中心室并全自动分割心室内外膜。首先,结合梯度方向直方图特征和支持向量机的心室自动检测方法,自动确定分割模型中的初始轮廓;其次, 将心室分割任务纳入统计形变模型形状特征点对齐任务框架,通过比较不同外观纹理特征和激活图,包括传统手工设计的特征和利用深度学习自动学习的卷积特征,提出利用堆叠多级沙漏卷积网络建模心室外观的全局和局部信息,统一活动外观模型和局部受限模型的概率形式,采用反向组合梯度下降算法迭代优化分割结果,完成左心室轮廓的自动提取。然后,以医生手动勾勒的轮廓作为“金标准”,通过构造心室分割数据集以评价算法,且提出了扩充数据样本的方法来克服深度卷积网络过拟合问题,进行详尽实验讨论分析了基于不同层级的多级沙漏卷积网络对全局和局部纹理特征建模能力对分割效果的影响。实验结果表明,卷积模块允许网络提取专门用于指定任务的特征,并通过实验显示其优于手工设计的特征。该方法分割效果优于传统形状对齐方法,能够解决自动定位超声心动图中左心室的初始轮廓和弱边界自动分割的问题。

\section{研究展望}

虽然在实验的测量中,CNN获得了巨大的成功,但是,仍然还有很多工作值得进一步研究。首先,鉴于最近的CNN变得越来越深,它们也需要大规模的数据库和巨大的计算能力,来展开训练。人为搜集标签数据库要求大量的人力劳动。所以,大家都渴望能开发出无监督式的CNN学习方式。

同时,为了加速训练进程,虽然已经有一些异步的SGD算法,证明了使用CPU和GPU集群可以在这方面获得成功,但是,开放高效可扩展的训练算法依然是有价值的。在训练的时间中,这些深度模型都是对内存有高的要求,并且消耗时间的,这使得它们无法在手机平台上部署。如何在不减少准确度的情况下,降低复杂性并获得快速执行的模型,这是重要的研究方向。

其次,我们发现,CNN运用于新任务的一个主要障碍是:如何选择合适的超参数?比如学习率、卷积过滤的核大小、层数等等,这需要大量的技术和经验。这些超参数存在内部依赖,这会让调整变得很昂贵。最近的研究显示,在学习式深度CNN架构的选择技巧上,存在巨大的提升空间。

最后,关于CNN,依然缺乏统一的理论。目前的CNN模型运作模式依然是黑箱。我们甚至都不知道它是如何工作的,工作原理是什么。当下,值得把更多的精力投入到研究CNN的基本规则上去。同时,正如早期的CNN发展是受到了生物视觉感知机制的启发,深度CNN和计算机神经科学二者需要进一步的深入研究。
  
有一些开放的问题,比如,生物学上大脑中的学习方式如何帮助人们设计更加高效的深度模型?带权重分享的回归计算方式是否可以计算人类的视觉皮质等等。我们希望这篇文章不仅能让人们更好地理解CNN,同时也能促进CNN领域中未来的研究活动和应用发展。