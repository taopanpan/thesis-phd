\chapter{总结与展望}
\label{chap:Sum}

 在大多数作品中,使用深层网络是显示出对现有技术的改进。由于这些改进似乎在各种领域都是一致的,并且通常情况下,深度学习解决方案的发展相对比较直接,我们可以将其视为医学计算领域的一大进步。然而,仍然存在一个主要问题,那就是我们如何以及何时能够在性能上实现巨大的飞跃 - 相当于2012年大规模类别识别增加10%。我们是否提出了正确的问题并调查了正确的任务?我们是否使用足够强大的输入表示(例如,2D vs 3D)?我们是否需要努力为每项医疗任务获得真正的大数据,还是要将学习转化为足够的?在这个问题的论文(第二部分)中部分地讨论了这些问题和更多问题,并且大部分仍然是将来要回答的挑战。在文献中,我们可以通过深度网络发现无人监督以及监督学习。看来大多数作品实际上都是使用监督式学习。如果在医学领域,数据编号是一个关键因素,形式主义应该结合无人监督和监督两方面的好处,问题就产生了。很可能要利用手头注释无法获取或棘手的大数据,该领域需要更多地转向半监督学习和无监督学习。文献包含许多网络架构。 vari-
能力很大,可以在当前杂志的作品中看到。可能性包括选择已知架构,设计特定任务架构,融合架构等等。今后一个有趣的问题是,如果在ILSVRC 2015分类任务中使用152层并且表现得最好的非常深的残留网络[36]在医疗任务中也取得了良好的结果。深度学习的一个关键方面是它可以从中受益
大量的训练数据。计算机视觉方面的突破性成果是基于ImageNET数据集(http://www.image-net.org/)在ILSVRC challanges上获得的。这个数据集与本期特刊中的大多数培训和测试数据集相比是非常大的(数百万对100或1000)。如果构建类似的大型公开可用医学图像数据集,我们的社区可能会受益匪浅。这有几个原因,这是具有挑战性的。首先,它很难获得建设数据集的资金。其次,医疗成像数据的高质量注释需要稀缺且昂贵的医疗专业知识。第三,隐私问题比共享自然图像更难以共享医疗数据。第四,医疗成像应用的广度需要收集许多不同的数据集。尽管存在这些潜在的障碍,但我们看到在数据收集和数据共享方面取得了迅速的进展。许多公共数据集已经发布,今天的研究经常使用它们进行实验验证。例子包括VISCERAL和癌症成像档案(http://www.visceral.eu/和http://www.cancerimagingarchive.net/)。 Roth等人[13]和Shin等人。 [17]分析CT扫描中肿瘤淋巴结肿大的数据集,他们已经在癌症影像档案库[37]上公布。同一组已经在线提供了胰腺数据集[38]。

自2007年以来,组织挑战已成为习惯在MICCAI,ISBI和SPIEMicalical Imaging等医学影像会议上举办研讨会。这导致了大量的数据集和正在进行的基准研究,记录在网站http://www.grand-challenge.org/上。使用这些公共基准数据集比使用公共有明显的优势:挑战提供了要解决的任务的精确定义,并定义了一个或多个评估指标,以提供所提议的算法之间公平和标准化的比较。如果没有这样的标准化,通常难以比较不同方法对相同问题即使他们使用相同的数据集。本研究阐述了这一问题,其中三项研究(Anthimopoulos等[16],Shin等[17]和van Tulder等[18])使用相同的胸部CT扫描数据集和间质肺标注疾病模式[19],但他们都以不同的形式报告结果。在这个问题上的一项研究(Setio等人[12])已经结合IEEE ISBI会议组织了对肺结节检测的挑战(http://luna16.grand- challenge.org/),使用公开可用的LIDC / IDRI数据集,因此本问题中描述的系统可以直接与其他方法进行比较。去年我们看到了第一个大型挑战到主要平台上组织的医学图像分析,这些分析主要集中在其他机器学习应用上。 Kaggle组织了一次关于从彩色眼底图像检测和分期糖尿病视网膜病变的竞赛,其中661支队伍提交了结果,获得了100,000美元奖金,并提供了约80,000张图片(https://www.kaggle.com/c/diabetic-视网膜病变检测)。这个数据在本次特刊的一项研究中被使用(van Grinsven et al。[24])。最近,第二次医学图像分析竞赛使用MRI完成,以测量心脏体积并获得192个参赛队的射血分数,并获得20万美元的奖金(https://www.kaggle.com/c/second-annual-data-科学碗)。在这两场比赛中,顶级竞争者都使用了绿色神经网络。我们预计这两种趋势都将持续下去:挑战将使用更大的数据集,深度学习将成为性能最好的解决方案中的主导技术。在这种情况下,即将举行的一系列全球竞赛旨在提高各种基于成像的癌症筛查的准确性(coding4cancer.org),可能会对这一特殊问题的读者感兴趣。在线平台,比如用于比赛的平台,有多种用途。他们导致新的合作和联合解决方案,也可能是通过众包获取大量数据注释的最有效方式,Albarqouni等人的研究表明了这一点。 [31]在这个问题上。最后,我们感谢现任总编辑的指导,TMI办公室的帮助以及最重要的是作者和审稿人的巨大努力。本期特刊提供了一个快速移动的医疗图像处理领域的快照。我们希望您会喜欢它,并期待您对这个充满活力的领域的未来贡献。