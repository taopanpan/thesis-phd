\chapter{总结与展望}
\label{chap:Sum}

通过构建标准切面数据库,提出了一种基于深度卷积神经网络的超声心动图标准切面自动识别方法,该算法针对网络全连接层占有模型大部分参数的缺点,引入空间金字塔均值池化替代全连接层,获得更多的空间结构信息,利用全局空间金字塔均值池化方法进行微调迁移学习,并大大减少模型参数、降低过拟合风险,通过类别显著性区域将类似注意力机制引入模型可视化过程,详尽分析了数据规模对模型分类精度的影响,并对模型的可解释性和有效性进行了分析。

针对基于深度卷积神经网络的图像分类模型的可解释性问题,通过评估模型特征空间的潜在可表示性,提出一种用于改善理解模型特征空间的可视化方法。给定任何已训练的深度卷积网络模型,引入了通过激活最大化获得的图像可解释性的正则化方法,结合现有正则化方法提出空间金字塔分解方法,利用构建多层拉普拉斯金字塔主动提升目标图像特征空间的低频分量,结合多层高斯金字塔调整其特征空间的高频分量得到较优可视化效果。并通过限制可视化区域,提出利用类别显著性激活图技术加以压制上下文无关信息,可进一步改善可视化效果。该模型有效克服了原有可视化方法中由于不能主动调整高低频分量等原因造成的可视化图像语义重复和低效率等问题。

针对自动检测医学图像中指定目标时存在的问题,提出了一种基于深度学习自动检测目标位置和估计对象姿态的算法。该算法基于区域深度卷积神经网络和目标结构的先验知识,采用区域生成候选框网络、感兴趣区域池化策略,引入包括分类损失、边框位置回归定位损失和像平面内朝向损失的多任务损失函数,近似优化一个端到端的有监督定位网络,能快速地对医学图像中目标自动定位,有效地为下一步的分割和参数自动提取提供定位结果。并在超声心动图左心室检测中提出利用检测额外标记点:二尖瓣环、心内膜垫和心尖,能高效地对左心室朝向姿态进行估计。

针对特征表示的底层视觉任务:图像去噪和分割中存在的问题,我们提出了一个有监督多层残差卷积网络框架,结合不同损失函数学习端到端映射变换。输入是带噪声的图像和原图像,输出的是去噪后的图像。针对医学超声图像的对比度低、存在斑点噪声导致难以分割的问题,提出一种利用沙漏卷积神经网络特征的多尺度形状模型分割方法,自动定位经食道超声心动图中心室并全自动分割心室内外膜。首先,结合梯度方向直方图特征和支持向量机的心室自动检测方法,自动确定分割模型中的初始轮廓;其次, 将心室分割任务纳入统计形变模型形状特征点对齐任务框架,通过比较不同外观纹理特征和激活图,包括传统手工设计的特征和利用深度学习自动学习的卷积特征,提出利用堆叠多级沙漏卷积网络建模心室外观的全局和局部信息,统一活动外观模型和局部受限模型的概率形式,采用反向组合梯度下降算法迭代优化分割结果,完成左心室轮廓的自动提取。然后,以医生手动勾勒的轮廓作为“金标准”,通过构造心室分割数据集以评价算法,且提出了扩充数据样本的方法来克服深度卷积网络过拟合问题,进行详尽实验讨论分析了基于不同层级的多级沙漏卷积网络对全局和局部纹理特征建模能力对分割效果的影响。实验结果表明,卷积模块允许网络提取专门用于指定任务的特征,并通过实验显示其优于手工设计的特征。该方法分割效果优于传统形状对齐方法,能够解决自动定位超声心动图中左心室的初始轮廓和弱边界自动分割的问题。

看来大多数作品实际上都是使用监督式学习。如果在医学领域,数据编号是一个关键因素,形式主义应该结合无人监督和监督两方面的好处,问题就产生了。很可能要利用手头注释无法获取或棘手的大数据,该领域需要更多地转向半监督学习和无监督学习。文献包含许多网络架构。 能力很大,可以在当前杂志的作品中看到。可能性包括选择已知架构,设计特定任务架构,融合架构等等。今后一个有趣的问题是,如果在ILSVRC 2015分类任务中使用152层并且表现得最好的非常深的残留网络[36]在医疗任务中也取得了良好的结果。深度学习的一个关键方面是它可以从中受益
大量的训练数据。

计算机视觉方面的突破性成果是基于ImageNET数据集(http://www.image-net.org/)在ILSVRC challanges上获得的。这个数据集与本期特刊中的大多数培训和测试数据集相比是非常大的(数百万对100或1000)。如果构建类似的大型公开可用医学图像数据集,我们的社区可能会受益匪浅。这有几个原因,这是具有挑战性的。首先,它很难获得建设数据集的资金。其次,医疗成像数据的高质量注释需要稀缺且昂贵的医疗专业知识。第三,隐私问题比共享自然图像更难以共享医疗数据。第四,医疗成像应用的广度需要收集许多不同的数据集。尽管存在这些潜在的障碍,但我们看到在数据收集和数据共享方面取得了迅速的进展。许多公共数据集已经发布,今天的研究经常使用它们进行实验验证。例子包括VISCERAL和癌症成像档案(http://www.visceral.eu/和http://www.cancerimagingarchive.net/)。 Roth等人[13]和Shin等人。 [17]分析CT扫描中肿瘤淋巴结肿大的数据集,他们已经在癌症影像档案库[37]上公布。同一组已经在线提供了胰腺数据集[38]。

自2007年以来,组织挑战已成为习惯在MICCAI,ISBI和SPIEMicalical Imaging等医学影像会议上举办研讨会。这导致了大量的数据集和正在进行的基准研究,记录在网站http://www.grand-challenge.org/上。使用这些公共基准数据集比使用公共有明显的优势:挑战提供了要解决的任务的精确定义,并定义了一个或多个评估指标,以提供所提议的算法之间公平和标准化的比较。如果没有这样的标准化,通常难以比较不同方法对相同问题即使他们使用相同的数据集。本研究阐述了这一问题,其中三项研究(Anthimopoulos等[16],Shin等[17]和van Tulder等[18])使用相同的胸部CT扫描数据集和间质肺标注疾病模式[19],但他们都以不同的形式报告结果。在这个问题上的一项研究(Setio等人[12])已经结合IEEE ISBI会议组织了对肺结节检测的挑战(http://luna16.grand- challenge.org/),使用公开可用的LIDC / IDRI数据集,因此本问题中描述的系统可以直接与其他方法进行比较。去年我们看到了第一个大型挑战到主要平台上组织的医学图像分析,这些分析主要集中在其他机器学习应用上。 

Kaggle组织了一次关于从彩色眼底图像检测和分期糖尿病视网膜病变的竞赛,其中661支队伍提交了结果,获得了100,000美元奖金,并提供了约80,000张图片(https://www.kaggle.com/c/diabetic-视网膜病变检测)。这个数据在本次特刊的一项研究中被使用(van Grinsven et al。[24])。最近,第二次医学图像分析竞赛使用MRI完成,以测量心脏体积并获得192个参赛队的射血分数,并获得20万美元的奖金(https://www.kaggle.com/c/second-annual-data-科学碗)。在这两场比赛中,顶级竞争者都使用了绿色神经网络。我们预计这两种趋势都将持续下去:挑战将使用更大的数据集,深度学习将成为性能最好的解决方案中的主导技术。在这种情况下,即将举行的一系列全球竞赛旨在提高各种基于成像的癌症筛查的准确性(coding4cancer.org),可能会对这一特殊问题的读者感兴趣。在线平台,比如用于比赛的平台,有多种用途。他们导致新的合作和联合解决方案,也可能是通过众包获取大量数据注释的最有效方式,Albarqouni等人的研究表明了这一点。 [31]在这个问题上。最后,我们感谢现任总编辑的指导,TMI办公室的帮助以及最重要的是作者和审稿人的巨大努力。本期特刊提供了一个快速移动的医疗图像处理领域的快照。我们希望您会喜欢它,并期待您对这个充满活力的领域的未来贡献。
深度CNN在图像处理、视频、语音和文本中取得了突破。本文种,我们主要从计算机视觉的角度对最近CNN取得的进展进行了深度的研究。我们讨论了CNN在不同方面取得的进步:比如,层的设计,活跃函数、损失函数、正则化、优化和快速计算。除了从CNN的各个方面回顾其进展,我们还介绍了CNN在计算机视觉任务上的应用,其中包括图像分类、物体检测、物体追踪、姿态估计、文本检测、视觉显著检测、动作识别和场景标签。

虽然在实验的测量中,CNN获得了巨大的成功,但是,仍然还有很多工作值得进一步研究。首先,鉴于最近的CNN变得越来越深,它们也需要大规模的数据库和巨大的计算能力,来展开训练。人为搜集标签数据库要求大量的人力劳动。所以,大家都渴望能开发出无监督式的CNN学习方式。

同时,为了加速训练进程,虽然已经有一些异步的SGD算法,证明了使用CPU和GPU集群可以在这方面获得成功,但是,开放高效可扩展的训练算法依然是有价值的。在训练的时间中,这些深度模型都是对内存有高的要求,并且消耗时间的,这使得它们无法在手机平台上部署。如何在不减少准确度的情况下,降低复杂性并获得快速执行的模型,这是重要的研究方向。
  
其次,我们发现,CNN运用于新任务的一个主要障碍是:如何选择合适的超参数?比如学习率、卷积过滤的核大小、层数等等,这需要大量的技术和经验。这些超参数存在内部依赖,这会让调整变得很昂贵。最近的研究显示,在学习式深度CNN架构的选择技巧上,存在巨大的提升空间。
  
最后,关于CNN,依然缺乏统一的理论。目前的CNN模型运作模式依然是黑箱。我们甚至都不知道它是如何工作的,工作原理是什么。当下,值得把更多的精力投入到研究CNN的基本规则上去。同时,正如早期的CNN发展是受到了生物视觉感知机制的启发,深度CNN和计算机神经科学二者需要进一步的深入研究。
  
有一些开放的问题,比如,生物学上大脑中的学习方式如何帮助人们设计更加高效的深度模型?带权重分享的回归计算方式是否可以计算人类的视觉皮质等等。
  我们希望这篇文章不仅能让人们更好地理解CNN,同时也能促进CNN领域中未来的研究活动和应用发展。